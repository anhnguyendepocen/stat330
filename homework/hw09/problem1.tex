  \item
    To assess whether a laboratory scale is accurate we can take a standard
    weight known to weigh exactly 100 grams and weigh it repeatedly.  Suppose
    that the scale reading has standard deviation
    $\sigma$= 0.3 grams. If the scale is accurate then the population mean
    $\mu$ (the mean obtained in many repeated weighings) would be 100 grams but
    if the scale is inaccurate the population mean could be higher or lower.  
    The weight is weighed 35 times and the sample mean is $\bar{x}$ =
        99.90.
    \begin{enumerate}	 
      \item Construct a 99\% confidence interval for $\mu$.
      \item Construct a 90\% confidence interval for $\mu$.
      \item Using part (b), comment on the hypothesis
        $H_0: \mu=100$.
      \item Test the hypothesis $H_0: \mu=100$ at level $a=0.1$.
    \end{enumerate}  
    
    
    \ansfont{
      \begin{enumerate}
        \item Using the large sample C.I. for population mean, the interval
          endpoints are
          \[\bar{x} \pm z_{\alpha/2} \frac{\sigma}{\sqrt{n}}.\]
        Since $z_{0.0025}=2.58$, the C.I. is $(99.7692, 100.0308)$.
        \item Using the large sample C.I. for population mean, the interval
          endpoints are
          \[\bar{x} \pm z_{\alpha/2} \frac{\sigma}{\sqrt{n}}.\]
          Since $z_{0.05}=1.65$, the C.I. is $(99.8163, 99.987)$.
        \item Since the C.I. in part (b) does not cover 100. Thus,
          there is a significant evidence that $H_0$ is not correct.
        \item This is a large sample scenario. The test statistic is
          \[
            Z = \frac{\bar{x}-100}{\sigma/\sqrt{n}} = -1.97
          \]
          Since it is a two-sided test, the $p$-value is
          \[
            P(|Z| \ge 1.97) = P(Z\le -1.97) + P(Z\ge1.97) = 2(0.0244) = 0.0488 < 0.1
          \]
          Thus, we reject the null hypothesis $H_0$ at level 0.1.
      \end{enumerate}
    }
