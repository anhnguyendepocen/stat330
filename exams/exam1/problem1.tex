%!TEX root = exam1.tex

\item The events $A_1$, $A_2$, $A_3$, and $A_4$ are a partition (or cover) of the sample space $\Omega$.

\begin{enumerate}

\item What is $P(A_1 \cap A_2)$? Why? (4 pts)

\ansfont{
0 because they are a partition and therefore disjoint
}
\vfill

\item If their individual probabilities are each 0.25, what is $P(\overline{A_1 \cup A_2 \cup A_3})$? (8 pts)

\ansfont{
\[ \begin{array}{rl}
P(\overline{A_1 \cup A_2 \cup A_3}) &= 1-P(A_1 \cup A_2 \cup A_3) \\
&= 1-P(A_1)-P(A_2)-P(A_3) \\
&= 1-0.25-0.25-0.25 \\
&= 0.25 
\end{array} \] 

alternatively, since $\overline{A_1 \cup A_2 \cup A_3} = A_4$
\[
P(\overline{A_1 \cup A_2 \cup A_3}) = P(A_4) = 0.25
\]
}
\vfill \vfill

\item If their individual probabilities are each 0.5, describe why this violates Kolmogorov's Axioms and therefore is not a valid probability model. (8 pts)

\ansfont{
Since these events are a partition, $A_1\cup A_2 \cup A_3 \cup A_4 = \Omega$. 
Axiom (ii) states that $P(\Omega)=1$, but by axiom (iii)
\[ P(\Omega) = P(A_1 \cup A_2 \cup A_3 \cup A_4) = P(A_1)+P(A_2)+P(A_3)+P(A_4) = 2 \]
}
\vfill \vfill



\end{enumerate}
