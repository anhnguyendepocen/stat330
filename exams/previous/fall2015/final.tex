\documentclass[12pt]{article}
\usepackage{amsmath, amssymb, amsthm, chicago, color, graphics, graphicx}

\setlength{\textwidth}{6.8in}
\setlength{\textheight}{9.9in}
\topmargin-1in
\evensidemargin-.15in
\oddsidemargin-.15in

\newcommand{\ansfont}[1]{{\textcolor{blue}{\textbf{Answer:}}\ \ #1}}

% Uncomment the following line to remove answers, and comment line out to show answers:
%\renewcommand{\ansfont}[1]{}





\begin{document}
\begin{center}
  \textbf{\large Spring 2016} \hfill \textbf{\large Statistics 330} \hfill \textbf{\large Final Exam}\\
  \hfill \textbf{(100 points)}
\end{center}
\vspace*{1in}
\noindent {\textbf{Name}} \rule{5in}{.01in}\\[1in]
\textbf{Instructions:}\\
\begin{itemize}
\item {\large Partial credit will be given only if you show your work.}
\item {\large The questions are not necessarily ordered from easiest to hardest.}
\item {\large You may only use your formula sheets (2 pages, both sides) and a calculator.}
\item {\large You can find z- and t-tables attached to the back of the exam.}
\item {\large You do not need to hand in your formula sheets along with your exam.}
\end{itemize}





\begin{enumerate}
\newpage % PROBLEM 1
\item Let $c_1, \ldots, c_k$ denote the $k$ components in a parallel system. Assume the $k$ components operate independently, and denote $P(c_j \text{ works})$ as $p_j$. Derive the formula for the reliability of the system in terms of the probabilities $p_j$. Clearly state which axioms, definitions, or results you are using at each stage of the development.

Note: Giving only the formula here will result in \emph{no credit}; you need to show that you understand where the formula comes from and how it is derived from basic facts about probabilities.



\ansfont{
}
\vfill





\newpage % PROBLEM 2
\item The \emph{Pareto distribution}, with parameters $\alpha$ and $\beta$, has pdf:
\[ f(x) = \frac{\beta \alpha^{\beta}}{x^{\beta + 1}} \text{ for } \alpha, \beta > 0 \]
The image of a random variable with this distribution is $(\alpha, \infty)$.
\begin{enumerate}
\item Verify that $f(x)$ is a pdf for any positive values of $\alpha$ and $\beta$.



\ansfont{
}
\vfill

\item Derive the mean and variance of this distribution when $\beta > 2$.



\ansfont{
}
\vfill
\end{enumerate}





\newpage % PROBLEM 3
\item Customers come to a bank teller's window according to a Poisson process at a rate of 1 customer every 15 minutes. Service times are independently exponentially distributed, with an average service time of 6 minutes. Customers that come while the teller is busy need to wait in line until the teller is free. There is only a single teller, and there is no limit to the length of the line.
\begin{enumerate}
\item Determine the average amount of time a customer spends waiting in line.



\ansfont{
}
\vfill

\item What fraction of the time is the teller busy with a customer?



\ansfont{
}
\vfill

\item What is the probability that an individual entering the bank needs to wait more than 10 minutes before being served?



\ansfont{
}
\vfill
\end{enumerate}





\newpage % PROBLEM 4
\item Thomas the Bayesian is fitting the simple linear regression model to some data $(x_i, y_i)$, for $i = 1, \ldots, n$. He has been told that the slope and the intercept of the regression line are 2 and -3, respectively, but he would like an estimate of $\sigma^2$ as well.

\vspace{12pt}

Thomas is considering using the improper prior $\pi(\sigma^2) = \left( \frac{1}{\sqrt{\sigma^2}} \right)^p$ to represent his beliefs. Derive the posterior distribution of $\sigma^2$ using this prior; it should have the form of one of the common ``named probability distributions'' seen in class. Which values of $p$ ensure that both parameter values are greater than 0?



\ansfont{
}
\vfill





\newpage % PROBLEM 5
\item Pigs are sent to a slaughterhouse according to a Poisson process with an unknown rate $\lambda$ pigs per minute.
\begin{enumerate}
\item What distributions do the inter-arrival times $I_1, I_2, \ldots$ follow? Are they independent?



\ansfont{
}
\vfill

\item You arrive at time $t = 0$ and see pigs enter the facility at times $t = 2$, 3, 4.5, and 7 minutes. Calculate the inter-arrival times and find the MLE of $\lambda$.



\ansfont{
}
\vfill
\end{enumerate}





\newpage % PROBLEM 6
\item CornCorp spends a lot of money creating new kinds of genetically modified corn. They have recently developed a new variety called ``supercorn'', and are now interested in testing how it compares to ``normal corn''.

CornCorp scientists have grown 1000 plants for each of the two types of corn, measured their heights (in feet), and have summarized their results as follows:
\begin{table}[h]
  \centering
  \begin{tabular}{lll}
    & mean height & sample variance \\
    normal corn & 8 & $\frac{1}{4}$ \\
    supercorn & 11 & $\frac{1}{2}$ \\
  \end{tabular}
\end{table}

\begin{enumerate}
\item We can imagine these 1000 supercorn plants as being a sample drawn from some infinite population of supercorn plants with mean height $\mu$. Create a 98\% confidence interval for $\mu$.



\ansfont{
}
\vfill

\item Is there enough evidence in our data to conclude that their is indeed a difference between the mean supercorn height $\mu$ and the mean normal corn height $\eta$? Using a type 1 error rate of $0.05$, determine this using either a hypothesis test or a confidence interval.



\ansfont{
}
\vfill
\end{enumerate}





\newpage % PROBLEM 7
\item Suppose we have some data $(x_i, y_i)$, for $i = 1, \ldots, n$ and we fit the simple linear regression model to these data using the least squares estimates for $\beta_0$ and $\beta_1$ and $\frac{1}{n - 2} \text{SSE}$ as our estimate for $\sigma^2$. That is, we are \textbf{not} putting a prior distribution on any of the parameters; we are estimating them using frequentist methods.

\begin{enumerate}
\item Describe the differences between $Var(\beta_0)$, $Var(\hat{\beta}_0)$, and $\hat{Var}(\hat{\beta}_0)$. Which one is used to create confidence intervals for $\beta_0$? Can the others be used instead?



\ansfont{
}
\vfill

\item Suppose we have a new observation with $x$-value $x_{new}$. Justify mathematically and intuitively why the prediction interval for $Y_{new}$ will be wider than the confidence interval for $E[\hat{Y}_{new}] = \beta_0 + \beta_1 x_{new}$.



\ansfont{
}
\vfill
\end{enumerate}

\end{enumerate}
\end{document}
