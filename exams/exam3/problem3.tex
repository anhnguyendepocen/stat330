%!TEX root = exam3.tex


\item In the most recent Bloomberg poll before the election, of the 800 
respondents 44\% of the respondents identified Clinton as their top choice 
while 41\% identified Trump as their choice. 

\begin{enumerate}
\item Provide an approximate 95\% confidence interval for the true percentage of 
Clinton supporters. (7 points)

\ansfont{
Let $Y_C$ be the number of Clinton supporters of the $n=800$ respondents and 
assume $Y_C\sim Bin(n,\theta_C)$. Let $\hat{\theta}_C = Y_C/n$ and 
$\hat{\theta}_C = 0.44$. Then an approximate 95\% confidence interval is 
\[ 
\frac{Y_C}{n} \pm z_{a/2} \sqrt{\frac{\hat{\theta}_C(1-\hat{\theta}_C}{n}} = 
0.44 \pm 1.96 \cdot \sqrt{\frac{0.44\cdot 0.56}{800}} =
(0.41,0.47)
\]
}
\vfill
% \vfill
% 
% \item Explain why we cannot calculate an approximate $100(1-a)$\% confidence 
% interval for the difference in proportions using the following formula. (3 pts)
% \[
% 0.44 - 0.41 \pm z_{a/2} \sqrt{\frac{0.44(1-0.44)}{800}+\frac{0.41(1-0.41)}{800}}
% \]
% 
% \ansfont{
% This formula is based on independence between the two samples but these responses
% are from the same poll and are therefore not independent. In particular,
% a respondent who indicating Clinton as their top choice could not indicate Trump
% as their top choice.
% }
% \vfill

\item Let $Y_C$ and $Y_T$ be the number of respondents who indicated Clinton 
and Trump as their top choices, respectively. Then $Y_C \sim Bin(n,\theta_C)$ 
and $Y_T \sim Bin(n,\theta_T)$. Using the fact that 
$Cov(Y_C,Y_T) = -n\theta_C \theta_T$, show that 
\[
Var\left( \frac{Y_C}{n} - \frac{Y_T}{n}\right) = 
\frac{1}{n}\left[ \theta_C(1 - \theta_C) + \theta_T(1-\theta_T) + 2\theta_C\theta_T \right]. \quad \mbox{(7 pts)}
\]

\ansfont{
\[ \begin{array}{rl}
Var\left( \frac{Y_C}{n} - \frac{Y_T}{n}\right) 
&= \frac{1}{n^2}Var(Y_C-Y_T) \\
&= \frac{1}{n^2}\left[ Var(Y_C)+Var(Y_T)-2Cov(Y_C,Y_T) \right] \\
&= \frac{1}{n^2}\left[ n\theta_C(1-\theta_C)+n\theta_T(1-\theta_T)+2n\theta_C \theta_T \right] \\
&= \frac{1}{n}\left[ \theta_C(1 - \theta_C) + \theta_T(1-\theta_T) + 2\theta_C\theta_T \right]
\end{array} \]
}

\vfill
% \vfill

\item Using the result in part (b), construct an approximate 95\% confidence interval for 
the true difference in Clinton versus Trump $(\theta_C-\theta_T$). (6 pts)

\ansfont{
Note that $\hat\theta_C = Y_C/n$ and $\hat\theta_T = Y_T/n$ and thus 
\[ 
\widehat{\theta_C-\theta_T} = \hat\theta_C-\hat\theta_T 
\quad
\mbox{with estimated standard error}
\quad
\sqrt{\frac{1}{n}\left[ \hat\theta_C(1 - \hat\theta_C) + \hat\theta_T(1-\hat\theta_T) + 2\hat\theta_C\hat\theta_T \right]}
\]
\[ \begin{array}{l}
\hat{\theta}_C - \hat{\theta}_T \pm z_{a/2} \sqrt{\frac{1}{n}\left[ \hat\theta_C(1 - \hat\theta_C) + \hat\theta_T(1-\hat\theta_T) + 2\hat\theta_C\hat\theta_T \right]} \\
=
0.44 - 0.41 \pm 1.96 \sqrt{\frac{1}{800}\left[ 0.44(1-0.44)+0.41(1-0.41)+2\cdot 0.44\cdot 0.41 \right]} \\
= 
% se = sqrt((0.44*(1-0.44)+0.41*(1-0.41)+2*0.44*0.41)/800)
% 0.44-0.41 + c(-1,1)*1.96*se
(-0.034, 0.094)
\end{array} \]
}
\vfill

\end{enumerate}
