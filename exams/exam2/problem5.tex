%!TEX root = exam2.tex

\item Fast food drive-thrus, where you order at the window, can be modeled as 
an $M/M/1$ queue. 
Suppose it takes, on average, 5 minutes to complete the order for a car and 
cars arrive at a rate of 10 per hour. 
For the following questions, assume the system is in its steady state 
distribution.

\begin{enumerate}
\item What proportion of the time is the drive-thru busy, i.e. there is at least
one customer? (7 points)

\ansfont{
The arrival rate is $\lambda_A = 10$ car/hour and the service rate is 
$\lambda_S = 60/5 = 12$ cars per hour. The utilization 
$r = \lambda_A/\lambda_S = 5/6$ is the proportion of time the drive-thru is 
busy, i.e. $P(X>0) = 5/6$ where $X$ is the number of people at the drive-thru
at steady state.
}
\vfill


\item On average, how many customers are waiting to place their order? (6 points)

\ansfont{
Let $X_w$ represent the length of the waiting line. 
\[ 
E[X_w] = \frac{r^2}{1-r} = \frac{(5/6)^2}{1-5/6} \approx 4.17
\]
So, on average, there are approximately 4.17 cars waiting to place their order.
}
\vfill

\item How long should a customer expect to wait before getting their food? 
(6 points)

\ansfont{
Let $R$ be the total time before getting their order, then 
\[ 
E[R] = \frac{1}{\lambda_S}\frac{1}{1-r} = \frac{1}{12} \frac{1}{1-5/6} =
\frac{6}{12} = 0.5
\]
So half an hour.  
}
\vfill

\end{enumerate}