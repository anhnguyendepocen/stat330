%!TEX root = exam2.tex

% From http://www.pstcc.edu/facstaff/rtjackso/math1530/classnotes/6_5.pdf

\item The weights of adult males are normally distributed with a mean of 172 
pounds and a standard deviation of 29 pounds (based on data from the National
Health Survey). 

\begin{enumerate}
\item What is the probability that one randomly selected adult male will weight 
more than 190 pounds? (7 points)

\ansfont{
Let $X_i\sim N(\mu,\sigma^2)$ be the weight of male $i$ with $\mu=172$ pounds 
and $\sigma=190$ pounds. 
Then 
\[ 
P(X_i > 190) = P\left(\frac{X_i-\mu}{\sigma} > \frac{190-172}{29} \right) 
\approx P(Z>0.62) = P(Z< -0.62) \approx 0.2676
\]
}
\vfill
\vfill

\item What is the probability that 25 randomly selected adult males will have a 
mean weight of more than 190 pounds? (7 points)

\ansfont{
Let $\overline{X}_n = \frac{1}{n}\sum_{i=1}^n X_i$, 
then $\overline{X}_n \sim N(\mu,\sigma^2/n)$

\[
P\left(\overline{X}_{25} > 190\right) = P\left(\frac{\overline{X}_{25}-\mu}{\sigma/\sqrt{n}} > \frac{190-172}{29/\sqrt{25}} \right) 
\approx P(Z>3.1) = P(Z< -3.1) \approx 0.0010
\]
}

\vfill
\vfill
\vfill

\item An elevator at a men's fitness center has a sign that says the maximum 
allowable weight is 4750 pounds. If 25 randomly selected men cram into the 
elevator, what is the probability it will be over the maximum allowable weight? 
(6 points)

\ansfont{
Let $S_n = \sum_{i=1}^n X_i = n\overline{X}_n$, 
then $S_n \sim N(n\mu,n\sigma^2)$

\[
P(S_{25} > 4750) = P\left(\overline{X}_{25} > 4750/25\right) = P\left(\overline{X}_n > 190\right)
\approx 0.0010
\]
from part b). 
}

\vfill
\vfill

\end{enumerate}