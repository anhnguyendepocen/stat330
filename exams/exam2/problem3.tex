%!TEX root = exam2.tex

% This is the CIP question which is calculate a 2-step transition probability
% for a 2x2 transition matrix


\item Suppose that if it snows on a given winter day in some town, then there is a 80\% chance that it will snow again the next day. If it does not snow one day, then there is a 40\% chance that it will snow the next day. The one step transition probability matrix is 


\[ P = \begin{blockarray}{ccc}
& S & N  \\
\begin{block}{c(cc)}
S & 0.8 & 0.2  \\
N & 0.4 & 0.6  \\
\end{block}
\end{blockarray}
\]



\begin{enumerate}
\item If it does not snow on Monday, what is the probability of it snowing on Wednesday? (8 pts)

\ansfont{ 
The 2-step transition probability matrix is

\[P^{(2)}=P^2=
 \begin{blockarray}{ccc}
& S & N  \\
\begin{block}{c(cc)}
S & 0.72 & 0.28  \\
N & 0.56 & 0.44  \\
\end{block}
\end{blockarray}
\]
 From the 2-step transition matrix, the probability is 0.56. 
}
\vfill
\vfill

\item Does this Markov chain have a steady-state distribution? Why or why not? (4 points)

\ansfont{ Because all one-step (or 2-step) transition probabilities are greater than zero, the steady state distribution must exist. (The Markov chain is regular.)}
\vfill

\item Find the steady state probability of snow on any winter day in this town.  (8 pts)
\ansfont{
Solve $\pi P = \pi$ with $\pi_S+\pi_N=1$:
  \[
    \begin{cases}
      0.8 \pi_S + 0.4 \pi_N = \pi_S\\
      0.2\pi_S + 0.6\pi_N = \pi_N\\
      \pi_S+ \pi_N=1
    \end{cases}
    \implies
    \begin{cases}
      \pi_S=2\pi_N\\
      \pi_S+ \pi_N=1
    \end{cases}
    \implies \pi_S=2/3, \quad \pi_N=1/3
  \]
  Thus the probability of snow is 2/3.
  }
\vfill
\vfill
\end{enumerate}
