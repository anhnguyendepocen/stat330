%!TEX root = exam2.tex

% This is the CIP question which is calculate a 2-step transition probability
% for a 2x2 transition matrix


\item Suppose that if it snows on a winter day in a town, then there is a 80\% chance that it will snow again the next day. If it does not snow one day, then there is a 40\% chance that it will snow the next day. 


\begin{enumerate}

\item Write the 1-step transition probability matrix. (4 pts)

\ansfont{ 
\[ P = \begin{blockarray}{ccc}
& S & N  \\
\begin{block}{c(cc)}
S & 0.8 & 0.2  \\
N & 0.4 & 0.6  \\
\end{block}
\end{blockarray}
\]
}


\vfill

\item Find the 2-step transition probability matrix. (4 pts)

\ansfont{ 
\[P^{(2)}=P^2=
 \begin{blockarray}{ccc}
& S & N  \\
\begin{block}{c(cc)}
S & 0.72 & 0.28  \\
N & 0.56 & 0.44  \\
\end{block}
\end{blockarray}
\]
}

\vfill

\item If it does not snow on Monday, what is the probability of it snowing on Wednesday? (2 pts)
\ansfont{ From the 2-step transition matrix, this probability is 0.56. 
}
\vfill


\item Use your answers to (a) and/or (b) to explain why this Markov chain must have a steady state distribution. (3 pts)

\ansfont{ Because all one-step (or 2-step) transition probabilities are greater than zero, the steady state distribution must exist. (The Markov chain is regular.)}
\vfill

\item Find the steady state probability of snow on any winter day in this town.  (7 pts)
\ansfont{
Solve $\pi P = \pi$ with $\pi_1+\pi_2=1$:
  \[
    \begin{cases}
      0.8 \pi_1 + 0.4 \pi_2 = \pi_1\\
      0.2\pi_1 + 0.6\pi_2 = \pi_2\\
      \pi_1+ \pi_2=1
    \end{cases}
    \implies
    \begin{cases}
      \pi_1=2\pi_2\\
      \pi_1+ \pi_2=1
    \end{cases}
    \implies \pi_1=2/3, \quad \pi_2=1/3
  \]
  Thus the probability of snow is 2/3.
  }
\vfill
\vfill

\end{enumerate}
