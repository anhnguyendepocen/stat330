\documentclass[10pt]{article}
\usepackage{graphicx, fancyhdr, enumerate}
\usepackage{amsmath, amsfonts, color}

\usepackage{hyperref}

\setlength{\topmargin}{-.5 in} 
\setlength{\textheight}{8.875 in}
\setlength{\textwidth}{6.5 in} 
\setlength{\evensidemargin}{0 in}
\setlength{\oddsidemargin}{0 in} 
\setlength{\parindent}{0 in}

%\rhead{\raisebox{0pt}{Name:\hspace{1in}}} 
\lhead{\Large\sffamily Stat 330A (Fall 2016): Syllabus} 
\rhead{\sffamily Last update: \today}
\cfoot{\thepage} 
\renewcommand{\headrulewidth}{0.4pt}
\renewcommand{\footrulewidth}{0pt} 
\newcommand{\sep}{\vspace*{0.4cm}}
\newcommand{\tab}{\hspace*{0.8cm}}

\begin{document}
\pagestyle{fancy}
%\layout
%\thispagestyle{empty}

%STAT 330. Probability and Statistics for Computer Science.
%
%(3-0) Cr. 3. F.S. Prereq: Math 166
%Topics from probability and statistics applicable to computer science. Basic probability; Random variables and their distributions; Elementary probabilistic simulation; Queuing models; Basic statistical inference; Introduction to regression. Nonmajor graduate credit.

%MTWRF 9:50am-10:50am 	MOL-BIO 1420 	LANKER CORY LEE
%	Meets: 05/14 - 07/06 

%PROBABILITY+STATISTICS F/COMP.SCIENTIST
%Author 	BARON
%ISBN 	9781584886419

%\vspace*{-3pc}
\begin{center}
\textbf{\large Stat 330A: Probability and Statistics for Computer Science}
\end{center}
\sep

\textbf{Lecture:} MWF 12:10--1:00 (2019 Morrill Hall)
\sep

\textbf{Instructor:} 
Jarad Niemi\\
\tab Email: {\tt niemi@iastate.edu}\\
\tab Office hours: TBD @ 2208 Snedecor Hall
\sep

\textbf{Teaching assistant:}
Xingche ``Michael'' Guo\\
\tab Email: {\tt xguo@iastate.edu}\\
\tab Office hours: TBD
\sep

\textbf{Prerequisites:} Math 166: Calculus II. 
%Students who have not taken this course \textbf{cannot}
%take this class as a departmental rule. 
%This prerequisite is required due to
%the calculus in the course.
\sep

\textbf{Course description:}
Topics from probability and statistics applicable to computer science. 
Basic probability; Random variables and their distributions; 
Stochastic processes including Markov chains; Queuing models; 
Basic statistical inference; Introduction to regression.
\sep

\textbf{Learning outcomes:}
\begin{enumerate}
  \item Students will acquire knowledge of topics in probability
    including rules of probability, counting methods, Bayes'~rule, random variables,
    expectation and variance, and common discrete and continuous distributions.
    Students will be able to apply these knowledge to some engineering
    applications such as computing system reliability.
  \item Students will be acquainted with stochastic processes including Markov processes and be able to
    conduct basic analysis, e.g.~computing average waiting time, 
    of queuing systems such as M/M/1 and M/M/k queue.
  \item Students will know fundamental methods for statistical inference such as descriptive statistics, confidence intervals and hypothesis tests.
  \item Students will learn about simple linear regression and be able to
    compute the least square estimates.
  \end{enumerate}
  \sep

\textbf{Text and references:} 
The required textbook for this class is \textit{Probability and Statistics for
  Computer Scientists, Second Edition (2013)}, by Michael Baron. We will cover most of chapters 1--4 and 6--11. 
%Note that this is the \underline{second edition} and the homework
%questions selected from this edition will not match the first edition of the book.

Lecture notes prepared by Professor Heike Hofmann will be provided on Blackboard.
The slide sets that I will use in the lecture are based on the textbook and
Professor Hofmann's notes.
\sep

\textbf{Course webpage:} Blackboard ({\tt bb.its.iastate.edu})
has course information, slide sets, assignments,
solutions, exam preparation materials, and access to grades.
%Checking your Iowa State e-mail on a daily basis is necessary as course communication
%through Blackboard uses your university e-mail account.
\sep

\textbf{Assessment:} Your final course grade is computed based on the following proportions:
30\% homework, 40\% in-semester exams and 30\% final exam.
\sep

\newpage
\tab \textbf{Homework:}
Homework assignments will be assigned throughout the semester with one assignment being dropped.
Homework is due at the \underline{beginning} of lecture on the due date.
There is a 20\% penalty for assignments that are instead turned in by the end of lecture.
Late homework submissions are given half-credit and are only accepted until the lecture following the due date. 
For each homework, \underline{only three random problems will be graded}.
Homework is to be completed and submitted individually. Group work is encouraged but all answers must be unique to the student.
\sep

\tab \textbf{In-semester exams:}
There will be three in-semester exams. Only the best two will be counted towards the final course grade (i.e. each of the two will be worth 20\% of your final course grade).
The tentative dates for these exams are September~23 (Friday), October~21
(Friday), and November~16 (Wednesday), but these dates are subject to change. 
The changes, if any, will be announced at least one week before the exam dates.
Each exam is 50 minutes, completed during the exam period.
%If the date of an exam changes,
%I will give at least one week's notice and make sure the change is announced in lecture
%and posted on Blackboard.
Since one in-semester exam will be dropped, I do not give make-up exams.
\sep

\tab \textbf{Final exam:}
The University has tentatively set the final exam on December~19 (Friday)
from 9:45--11:45am. Note the date and time are subject to change.
The final exam is cumulative and may not be taken early under any circumstances.
\sep

\tab \textbf{Academic Dishonesty, Disability Accommodation, Dead Week, Harassment and Discrimination, Religious Accommodation:} This course abides by the Faculty Senate Recommendations provided at \url{http://www.celt.iastate.edu/teaching/preparing-to-teach/recommended-iowa-state-university-syllabus-statements}.

\end{document}
