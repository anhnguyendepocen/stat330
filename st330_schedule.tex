\documentclass[10pt]{article}
\usepackage{graphicx, fancyhdr, enumerate}
\usepackage{amsmath, amsfonts, color}

\setlength{\topmargin}{-.5 in} 
\setlength{\textheight}{8.875 in}
\setlength{\textwidth}{6.5 in} 
\setlength{\evensidemargin}{0 in}
\setlength{\oddsidemargin}{0 in} 
\setlength{\parindent}{0 in}
\newcommand{\ben}{\begin{enumerate}}
    \newcommand{\een}{\end{enumerate}}
\newcommand{\xbar}{\overline{X}}
\newcommand{\ansfont}[2]{{\sffamily \color{blue} \textbf{Concepts: #1} #2}}
\newcommand{\correction}[1]{{\color{green} \emph{#1}}}
%\newcommand{\refs}[2]{Baron: #1 {\sffamily \color{blue} Slide sets: #2 }}
\newcommand{\refs}[2]{{\sffamily \color{blue} Slide sets: #2} Baron: #1}
%\newcommand{\refs}[2]{Slide sets: #2 Baron: #1}
\newcommand{\sepline}{{\color{blue}\noindent\rule{16.5cm}{0.4pt}}\addtocounter{section}{1}\setcounter{subsection}{0}}

%\rhead{\raisebox{0pt}{Name:\hspace{1in}}} 
\lhead{\Large\sffamily Stat 330 (Fall 2016): Teaching schedule} 
\rhead{\sffamily Last update: \today}
\cfoot{\thepage} 
\renewcommand{\headrulewidth}{0.4pt}
\renewcommand{\footrulewidth}{0pt} 

\def\Exp#1#2{\ensuremath{#1\times 10^{#2}}}
\def\Case#1#2#3#4{\left\{ \begin{tabular}{cc} #1 & #2 \phantom{\Big|}  #3 & #4 \phantom{\Big|} \end{tabular} \right.}
%part 	-1,chapter 	0,section 	1,subsection 	2
%\renewcommand{\subsection}{\@startsection
%{subsection}%                   % the name
%{2}%                         % the level
%{0mm}%                       % the indent
%{-.5\baselineskip}%            % the before skip
%{.25\baselineskip}%          % the after skip
%{\normalfont\large\itshape}} % the style
\def\thesubsection       {\thesection-\arabic{subsection}.\hspace{-6pt}}


\begin{document}
\pagestyle{fancy} 

Below is the tentative schedule for the semester with weekly topics as well as associated slide sets, sections from Probability and Statistics for Computer Science (2nd. ed), and sections from Professor Hoffman's notes.

\begin{center}
\begin{tabular}{|l|l|l|l|l|}
\hline
Week & Topic & Slide set & Book sections & Hoffman notes \\
\hline
1 & Experiments & 1 & 1 & 1.1 \\
& Probability & 2  & 2.1--2.2 & 1.2-1.3 \\
& Permutations and combinations & 3 & 2.3 & 1.4 \\
\hline
2 & Conditional probability & 4 & 2.4 & 1.5--1.6 \\
& Bayes' rule & 5 & 2.4 & 1.7 \\
\hline
3 & Discrete random variables & 6 & 3.1--3.3 & 2.1\\
& Discrete distributions & 7 & 3.4 & 2.2 \\
\hline
4 & Continuous random variables & 8 & 4.1 & 2.3 \\
& Continuous distributions & 9 & 4.2 & 2.4 \\
& Central limit theorem & 10 & 4.3 & 2.5 \\
\hline 
5 & Multiple random variables & 11 & 3.2 & 2.2.5 \\
\hline
\hline
\multicolumn{5}{|c|}{Mid-term I*} \\
\hline
\hline
6 & Stochastic processes & 12 & 6.1--6.2 & 4 \\
& Poisson process & 13 & 6.3 & 4.1 \\
\hline
7 & Queuing systems & 14 & 7.1--7.2 & 5.1--5.2 \\
 & M/M/1 & 15 & 7.4 & 5.2 \\
\hline
8 & M/M/k & 16 & 7.5 & 5.3 \\
& M/M/$\infty$ & 16 &  & 5.4 \\
\hline
9 &  &  & & \\
\hline
\hline
\multicolumn{5}{|c|}{Mid-term II*} \\
\hline
\hline
10 & Descriptive statistics &  & 8.1--8.2 & \\
& Graphical statistics &  & 8.3 & \\
\hline 
11 & Method of moments &  & 9.1 & \\
& Maximum likelihood estimation &  & 9.2 & 6.1 \\
& Confidence intervals &  & 9.3 & 6.2\\
\hline
12 & Bayesian estimation &  & 9.3 & 6.2\\
& Credible intervals & & & \\
& Posterior model probabilities & & & \\
\hline
13 & Hypothesis testing &  & 9.4 & 6.3 \\
\hline
\hline
\multicolumn{5}{|c|}{Mid-term III*} \\
\hline
\hline
14 & Simple linear regression &  & 11.1 & 6.4 \\
& &  & 11.2 & \\
\hline
15 & &&& \\
& & & & \\
\hline
\end{tabular}
\end{center}

* The coverage of exams will be announced a week before the exams.

\end{document}

