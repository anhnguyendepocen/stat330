\documentclass[20pt,landscape]{foils}
\usepackage{amsmath, amssymb, amsthm}
\usepackage{color}
\usepackage{hyperref}
%\usepackage{pause}
\usepackage{graphicx}
\usepackage{epsfig}
%\usepackage{geometry}
%\geometry{headsep=3ex,hscale=0.9}
\newcommand{\bd}{\textbf}
\newcommand{\no}{\noindent}
\newcommand{\un}{\underline}
\newcommand{\bi}{\begin{itemize}}
\newcommand{\ei}{\end{itemize}}
\newcommand{\be}{\begin{enumerate}}
\newcommand{\ee}{\end{enumerate}}
\newcommand{\bc}{\begin{center}}
\newcommand{\ec}{\end{center}}
\newcommand \h {\hspace*{.3in}}
\newcommand{\bul}{\hspace*{.3in}{\textcolor{red}{$\bullet$ \ }}}
\newcommand{\xbar}{\bar{x}}
\rightheader{Stat 330 (Fall 2016): slide set 2}

\begin{document}
\LogoOff

\foilhead[1.3in]{}
\centerline{\LARGE \textcolor{blue}{Slide set 2}}
\vspace{0.3in}
\centerline{\large Stat 330 (Fall 2016)}
\vspace{0.2in}
\centerline{\tiny Last update: \today}
\setcounter{page}{0}

\foilhead[-.7in]{\textcolor{blue}{Probability}}
%\vspace*{-.4in}
\no \textbf{Example:}
\bi
%\addtolength{\itemsep}{-0.6\baselineskip}
\item[\bul] Consider the Event C (a successful transmission) defined earlier. 

\item[\bul] From our experience with the network provider, we can decide that the 
chance that the next message gets through is 90 \%.

\item[\bul] We write: $P(C) = 0.9$

\item[\bul] To be able to work with probabilities, in particular, to be able to compute \textcolor{magenta}{probabilities of events},
a mathematical foundation is necessary.
\ei

\foilhead[-.7in]{\textcolor{blue}{Sets and Venn Diagrams}}
%\vspace*{-.4in}
\be
\addtolength{\itemsep}{-0.6\baselineskip}
\item  Review symbols $\in, \notin, \subset, \subseteq, \supset, \supseteq, $.\\[.05in]
For e.g.
\bi
\item[\bul] If a is a member of B, this is denoted $a  \in B$
\item[\bul] If every member of set A is also a member of set B, then A is said to be a subset of B, written $A \subseteq B$ (A is said to be contained in B).
\ei
\item \textcolor{magenta}{Union} ($\cup$): 
A union of events is an event consisting of all the outcomes in these events.\\[.1in]
\h \h \h $A\cup B = \{\omega\mid \omega \in A \text{ or } \omega \in B\}$
\item \textcolor{magenta}{Intersection} ($\cap$): 
An intersection of events is an event consisting of the common outcomes in these events.\\[.1in]
\h \h \h$A\cap B = \{\omega\mid \omega \in A \text{ and } \omega \in B\}$
\newpage
\item \textcolor{magenta}{Complement} ($\bar A$): 
A complement of an event $A$ is an event that occurs when event $A$ does not happen.\\[.1in]
\h \h \h \ $\bar A = \{\omega \mid \omega \notin A\}$

\item \textcolor{magenta}{Demorgan's Law}:\\[.1in]
\h \h \h  $(\overline{A\cup B}) = \bar A\cap \bar B$

\item \textcolor{magenta}{Empty Set} $\emptyset$ is a set having no elements, usually denoted by $\{\}$\\[.05in]
The empty set is a subset of every set:\\[.1in]
\h \h \h  $\emptyset \subseteq A$

\item \textcolor{magenta}{Disjoint sets}: Sets $A, B$ are disjoint if their intersection is empty:\\[.1in]
\h \h \h $ A\cap B = \emptyset$

\item \textcolor{magenta}{Mutually exclusive sets}: Sets $A_1, A_2, \ldots$ are
  mutually exclusive if any two of these events are disjoint:\\[.1in]
\h \h \h $ A_i\cap A_j  = \emptyset$ for any $i\neq j$.
\ee


\foilhead[-.7in]{\textcolor{blue}{Sets and Venn Diagrams (contd.)}}
\vspace*{.2in}
\bc
\includegraphics[scale=1.3]{fig2-1.eps}
\ec


\foilhead[-.5in]{\textcolor{blue}{Kolmogorov's Axioms}}
\no To be able to work with probabilities properly - to compute with 
them - one must lay down a set of postulates:

\no A system of probabilities (\textcolor{magenta}{a probability model}) is an assignment of 
numbers $P(A)$ to events $A \subset \Omega$ (requires further mathematical complications for the events that we can define probability on) in such a manner that
\vspace*{-0.5cm}
\begin{itemize}
    \item[(i)]  $0 \le P(A) \le 1$ for all $A$ 
    \item[(ii)] $P(\Omega) = 1$. 
    \item[(iii)]  if $A_{1},A_{2}, \ldots$ are (possibly, infinite many)
      mutually exclusive events 
    (i.e. $A_{i} \cap A_{j} = \emptyset$ for all $i\neq j$) then 
    \[
    P(A_{1} \cup A_{2} \cup \ldots ) = P(A_{1}) + P(A_{2}) + \ldots = 
    \sum_{i}P(A_{i}).
    \]
   
\end{itemize}



\foilhead[-.8in]{\textcolor{blue}{Kolmogorov's Axioms (continued) }}
\no These are the basic rules of operation of a probability model\\
\h \h \h \bul every valid model must obey these,\\
\h \h \h \bul any system that does, is a valid model\\[.01in]
\no Whether or not a particular model is realistic or appropriate for a 
specific application is another question.\\[.1in]
\no \textbf{Example:}\\
Draw a single card from a standard deck of playing cards:
$\Omega = \{ \textcolor{red}{red}, black \}$\\[.1in]
\no Two different, equally valid probability models are:
\begin{center}
{\small
\begin{tabular}{l@{\extracolsep{.5in}}l}
    \underline{Model 1} & \underline{Model 2} \\
%    $P(0) = 0$ & $P(0) = 0$ \\
    $P(\Omega) = 1$ & $P(\Omega) = 1$ \\
    $P(\textcolor{red}{red}) = 0.5$ &  $P(\textcolor{red}{red}) = 0.3$ \\
    $P(black) = 0.5$ &  $P(black) = 0.7$
\end{tabular}}
\end{center}
\vspace*{-.2in}
Mathematically, both schemes are equally valid. But, of course, our real world experience would favor to pick model 1 over model 2 as the `correct' model.

\foilhead[-.5in]{\textcolor{blue}{Useful Consequences of Kolmogorov's Axioms:}}
\noindent  Let $A,\,B\subset\Omega$. 
\begin{itemize}
\item[1.] Probability of the Complementary Event: 
$ P(\overline{A}) = 1-P(A)$\\[.1in]
\hspace*{.2in} {\textcolor{red}{Corollary:}} $P(\emptyset)=0$
\item[2.] Addition Rule of Probability \\[.1in]
\begin{minipage}[b]{3in}
    \mbox{}
    \[
    P(A \cup B) = P(A) + P(B) - P(A \cap B)
    \]
    \vfill
\end{minipage}
\hfill
\begin{minipage}[b]{4in}
    \mbox{}
    \centerline{\includegraphics[width=4in]{fig2-2.eps}}
\end{minipage}
\item[3.] If $A\subset B$, then $P(A)\leq P(B)$.\\[.1in]
\hspace*{.2in}  {\textcolor{red}{Corollary:}} For any $A$, $P(A)\leq 1$. 
\end{itemize}
\foilhead[-.5in]{\textcolor{blue}{Example: Using Kolmogorov's Axioms}}
We attempt to sign on to AOL using dial-up. We connect successfully if and only if the phone number works \emph{ and } the AOL network works. Assume 
\begin{eqnarray*}
P(\text{ phone up } ) &=& .9   \\ 
P(\text{ network up } ) &=& .6, \mbox{ and } \\ 
P(\text{ phone up and network up } ) &=& .55.
\end{eqnarray*}
\begin{enumerate}
\item What is the probability that the phone is up or the network is up?
\item What is the probability that both the phone and the network are down? 
\item What is the probability that we fail to connect?
\item What is the probability that only the phone is up?
\end{enumerate}
\foilhead[-.7in]{\textcolor{blue}{Solution}}
\no Let  $A\equiv \text{phone up};\ B\equiv \text{network up}$\\[.2in]
\no 1) What is the probability that the phone is up or the network is up?\\[.2in]
  $P( \text{ phone up or network up}) =P(A\cup B) = 0.9 + 0.6 - 0.55 = 0.95$\\[.3in]
\no 2)  \no What is the probability that both the phone and the network are down?\\[.2in]
 $P( \text{ phone down and network down})=P(\bar A\cap \bar B) = P(\overline{A\cup B})$\\[.3in]
 \hspace*{3in} $=1-.95=.05$\\[.2in]
 \no 3) What is the probability that we fail to connect?\\[.1in]
  $P( \text{ phone down or network down}) =P(\bar A\cup \bar B)=P(\bar A) + P(\bar B) -P(\bar A\cap \bar B)$\\[.3in]
\hspace*{2in} $P(\bar A\cup \bar B) =(1-.9)+(1-.6)-.05=.1+.4-.05=.45$


\end{document}






\section{Kolmogorov's Axioms}
Example:
\begin{enumerate}
    \item[3.]
From my experience with the network provider, I can decide that the 
chance that my next message gets through is 90 \%.

Write: $P(D) = 0.9$
\end{enumerate}

To be able to work with probabilities properly - to compute with 
them - one must lay down a set of postulates:

\paragraph*{Kolmogorov's Axioms}

A system of probabilities ( a probability model) is an assignment of 
numbers $P(A)$ to events $A \subset \Omega$ in such a manner that
\begin{itemize}
    \item[(i)]  \notiz{the probability of any event $A$ 
    is a real number between 0 and 1}
    $0 \le P(A) \le 1$ for all $A$ 
    \item[(ii)] \notiz{the sum of pro\-ba\-bi\-li\-ties of all 
    events in the sample space is 1}
    $P(\Omega) = 1$. 
    \item[(iii)]  \notiz{the pro\-ba\-bi\-lity of a disjoint union of events is equal to the 
    sum of the individual pro\-ba\-bi\-li\-ties}
    if $A_{1},A_{2}, \ldots$ are (possibly, infinite many) disjoint events 
    (i.e. $A_{i} \cap A_{j} = \emptyset$ for all $i, j$) then 
    \[
    P(A_{1} \cup A_{2} \cup \ldots ) = P(A_{1}) + P(A_{2}) + \ldots = 
    \sum_{i}P(A_{i}).
    \]
   
\end{itemize}

These are the basic rules of operation of a probability model:
\begin{itemize}
    \item every valid model must obey these,
    \item any system that does, is a valid model
\end{itemize}
Whether or not a particular model is realistic or appropriate for a 
specific application is another question.

\begin{xpl}{}
Draw a single card from a standard deck of playing cards
\[
\Omega = \{ \textcolor{ red, black } \}
\]
Two different, equally valid probability models are:

\begin{center}
\begin{tabular}{ll}
    \underline{Model 1} & \underline{Model 2} \\
%    $P(0) = 0$ & $P(0) = 0$ \\
    $P(\Omega) = 1$ & $P(\Omega) = 1$ \\
    $P(\textcolor{red}{red}) = 0.5$ &  $P(\textcolor{red}{red}) = 0.3$ \\
    $P(black) = 0.5$ &  $P(\textcolor{ black }) = 0.7$
\end{tabular}
\end{center}

Mathematically, both schemes are equally valid. But, of course, our real world experience would favor to pick model 1 over model 2 as the `correct' model.
\end{xpl}
This is a very important point: even though the math of a problem might be completely correct, it might not reflect the real world in any way. {\bf We} have to make sure that a model we pick describes a real world situation correctly, or at least as well as we can. 



Beginning from the axioms of probability one can prove a number of 
useful theorems about how a probability model must operate.

We start with the probability of $\Omega$ and derive others from that.

\begin{satz}\label{complement}
    Let $A$ be an event in $\Omega$, then 
    \[
    P(\bar{A}) = 1 - P(A) \textcolor{ for all } A \subset \Omega.
    \]
\end{satz}

\begin{bew}
    For the proof we need to consider three main facts and piece them 
    together appropriately:
    \begin{enumerate}
	\item We know that $P(\Omega) = 1$ because of axiom (ii)
	\item $\Omega$ can be written as $\Omega = A \cup \bar{A}$ because of 
	the definition of an event's complement.
	\item $A$ and $\bar{A}$ are disjoint and therefore the probability of 
	their union equals the sum of the individual probabilities (axiom iii).
    \end{enumerate}
    All together:
    \[
    1 \stackrel{(1)}{=} P(\Omega) \stackrel{(2)}{=} P(A \cup \bar{A}) 
    \stackrel{(3)}{=} P(A) + P(\bar{A}).
    \]
    This yields the statement.
\end{bew}

\begin{xpl}{}
\begin{itemize}
    \item[3.] If I believe that the probability that a message gets 
    through is 0.9, I also \em{must} believe that it fails with 
    probability 0.1
\end{itemize}
\end{xpl}

\begin{kor}
    The probability of the empty set $P(\emptyset)$ is zero.
\end{kor}
\begin{bew}
    For a proof of the above statement we exploit that the empty set 
    is the complement of $\Omega$. Then we can apply Theorem \ref{complement}.
    \[
    P(\emptyset) = P(\bar{\Omega}) \stackrel{\textcolor{Thm \ref{complement}}}{=} 
    1 - P({\Omega}) = 1 - 1 =0.
    \]
\end{bew}

\begin{xsatz}{Addition Rule of Probability}
\begin{minipage}[b]{3in}
    \mbox{}
    Let $A$ and $B$ be two events of $\Omega$, then:
    \[
    P(A \cup B) = P(A) + P(B) - P(A \cap B)
    \]
    
    \vfill
    
\end{minipage}
\hfill
\begin{minipage}[b]{2in}
    \mbox{}
    \centerline{\includegraphics[width=1.5in]{ps/addition}}
\end{minipage}
\end{xsatz}    
To see why this makes sense, think of probability as the area in the 
Venn diagram: By simply adding $P(A)$ and $P(B)$, $P(A \cap B)$ gets 
counted twice and must be subtracted off to get $P(A \cup B)$.

\begin{xpl}{}
\begin{enumerate}
    \item AOL dial-up:
    
    If I judge: 
    \begin{eqnarray*}
	P( \textcolor{ phone up }) &=& 0.9 \\
	P( \textcolor{ network up }) &=& 0.6 \\
	P( \textcolor{ phone up, network up }) &=& 0.55 \\
    \end{eqnarray*}
{\it    then
    
    $P( \textcolor{ phone up or network up}) = 0.9 + 0.6 - 0.55 = 0.95$
  
    diagram:
    
    \begin{center}
	\begin{tabular}{lr|cc|c}
	    && \multicolumn{2}{c}{phone} \\
	    && up & down & \\ \hline
	 network   & up & .55 & .05 & .60 \\
	    & down & .35 & .05 & .40 \\ \hline
	    && .90 & .10 & 1
	\end{tabular}
    \end{center}
    }
\end{enumerate}
\end{xpl}
%
\begin{xpl}{}
A box contains 4 chips, 1 of them is defective.

A person draws one chip at random.

What is a suitable probability that the person draws the defective 
chip?

{\it Common sense tells us, that since one out of the four chips is 
defective, the person has a chance of 25\% to draw the defective chip.

Just for training, we will write this down in terms of probability 
theory:

One possible sample space $\Omega$ is: $\Omega = \{ g_{1}, g_{2}, g_{3}, d \}$
(i.e. we distinguish the good chips, which may be a bit artificial. It 
will become obvious, why that is a good idea anyway, later on.) 

The event to draw the defective chip is then $A  = \{ d \}$.

We can write the probability to draw the defective chip by comparing 
the sizes of $A$ and $\Omega$:
\[
P(A) = \frac{|A|}{|\Omega|} = \frac{|\{d\}|}{|\{ g_{1}, g_{2}, g_{3}, d 
\}|} = 0.25.
\]
}
\end{xpl}

Be careful, though! The above method to compute probabilities is only 
valid in a special case:
\begin{satz}\label{equally.likely}
    If all elementary events in a sample space are equally likely (i.e. 
    $P(\{ \omega_{i} \}) = $ const for all $\omega \in \Omega$), the 
    probability of an event $A$ is given by:
    \[
    P(A) = \frac{|A|}{|\Omega|},
    \]
    where $|A|$ gives the number of elements in $A$.
\end{satz}

\begin{xpl}{{continued}}
The person now draws two chips. 
What is the probability that the defective chip is among them?

{\it 
We need to set up a new sample space containing all possibilities for 
drawing two chips:

\begin{eqnarray*}
\Omega &=& \{ \{g_{1}, g_{2}\},  \{g_{1}, g_{3}\}, \{g_{1}, d\}, \\
       &&  \ \  \{g_{2}, g_{3}\}, \{g_{2}, d\}, \\
       &&  \ \  \{g_{3}, d\} \} 
\end{eqnarray*}
\begin{eqnarray*}
E &=& \textcolor{ `` defective chip is among the two chips drawn'' }= \\
 &=& \{ \{g_{1}, d\}, \{g_{2}, d\}, \{g_{3}, d\} \}.
\end{eqnarray*}
Then
\[
P(E) = \frac{|E|}{|\Omega|} = \frac{3}{6} = 0.5.
\]
}
\end{xpl}

Finding $P(E)$ involves counting the number of outcomes in $E$. 
Counting by hand is sometimes not feasible if $\Omega$ is large.

Therefore, we need some standard counting methods.
