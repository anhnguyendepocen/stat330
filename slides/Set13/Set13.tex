\documentclass[20pt,landscape]{foils}
\usepackage{amsmath, amssymb, amsthm}
\usepackage{amstext}
\usepackage{amsgen}
\usepackage{amsxtra}
\usepackage{amsgen}
\usepackage{amsthm}
\usepackage{color}
\usepackage{hyperref}
%\usepackage{pause}
\usepackage{graphicx}
\usepackage{epsfig}
%\usepackage{geometry}
%\geometry{headsep=3ex,hscale=0.9}
\newcommand{\bd}{\textbf}
\newcommand{\no}{\noindent}
\newcommand{\un}{\underline}
\newcommand{\bi}{\begin{itemize}}
\newcommand{\ei}{\end{itemize}}
\newcommand{\be}{\begin{enumerate}}
\newcommand{\ee}{\end{enumerate}}
\newcommand{\bc}{\begin{center}}
\newcommand{\ec}{\end{center}}
\newcommand \h {\hspace*{.3in}}
\newcommand{\bul}{\hspace*{.3in}{\textcolor{red}{$\bullet$ \ }}}
\newcommand{\xbar}{\bar{x}}
\rightheader{Stat 330 (Fall 2016): slide set 13}

\begin{document}
\LogoOff

\foilhead[1.3in]{}
\centerline{\LARGE \textcolor{blue}{Slide set 13}}
\vspace{0.3in}
\centerline{\large Stat 330 (Fall 2016)}
\vspace{0.2in}
\centerline{\tiny Last update: \today}
\setcounter{page}{0}

\foilhead[-.75in]{\textcolor{blue}{Exponential Distribution}}\vspace{.01in}
\no This distribution is commonly used to model waiting times between 
occurrences of \emph{rare} events, lifetimes of electrical or mechanical 
devices.\\[.1in]
\no  {\textcolor{magenta}{Exponential density}} {\textcolor{cyan}{A random variable $X$ has {\it exponential density} if}}
  \vspace*{-.3in}
   
    \[
    f_{X}(x) = \left \{
    \begin{array}{cl}
	\lambda e^{-\lambda x} & \text{ if } x \ge 0 \\
	0 & \text{ otherwise}
    \end{array}
    \right .
    \]
  \vspace*{-.1in}  
\no {\textcolor{cyan}{$\lambda$ is called the {\it rate parameter}}}. We say that the random variable $X \sim Exp(\lambda)$\\[.15in]
\no Mean, variance and cumulative distribution function are easy to compute. They 
are:
\vspace*{-.2in}
\begin{eqnarray*}
    E[X] &=& \frac{1}{\lambda} \\
    Var[X] &=& \frac{1}{\lambda^{2}}\\
    Exp_{\lambda}(t) &=& F_{X}(t) = \left \{
        \begin{array}{cl}
	    0  & \text{ if } t < 0 \\
	    1 - e^{-\lambda t} & \text{ if } t \ge 0 \\
	\end{array}
    \right .
\end{eqnarray*}
 
 
\foilhead[-.7in]{\textcolor{blue}{Density functions of exponential variables for different 
    rate parameters 0.5, 1, and 2.}}\vspace{10pt}
    \centerline{\includegraphics[scale=.8]{exponential.pdf}}
    
 \foilhead[-.7in]{\textcolor{blue}{Example 4.5 (Baron)}}\vspace{.01in}
  \no {\textcolor{magenta}{Example:}}{\textcolor{cyan}{Jobs are sent to a printer  at an average of 3 jobs per hour. (a) What is the expected time between jobs? (b) What is the  probability that the next job is sent within 5 minutes?}}\\[.1in]
 \no {\textcolor{magenta}{Solution:}} Job arrivals represent rare events, thus the time $T$ between them is Exponential with rate 3 jobs/hour i.e. $\lambda=3$.\\[.1in]
\no \h (a) Thus $E(T)=1/\lambda= 1/3$ hours or $20$ minutes.\\[.1in]
\no \h (b) Using the same units (hours) we have 5 min.=1/12 hours. Thus we compute 
$$P(T<1/12) = Exp_{3}(1/12)=1-e^{-3 \cdot\frac{1}{12}}=1-e^{-\frac{1}{4}}=0.2212$$
    
\foilhead[-.7in]{\textcolor{blue}{Exponential Distribution: Example}}\vspace{.01in}    
\no {\textcolor{red}{Note: The following example will be continued throughout the remainder of this class.}}\\[.1in]
\no {\textcolor{magenta}{Example:}} {\textcolor{cyan}{Hits on a webpage}}\\[.1in]
\no Suppose we are told that, on average, there are 2 hits per minute on a specific web page. \\[.1in]
\no I start to observe this web page at a certain time point 0, and decide to model the waiting time till the first hit $Y$ ({\textcolor{magenta}{in minutes}}) using an {\textcolor{cyan}{exponential distribution}}.\\[.1in]
\no For modeling the distribution of $Y$ using the exponential, we need an appropriate value for $\lambda$, the rate parameter\\[.1in]
\no {\textcolor{red}{Since, on average there are 2 hits per minute, the average waiting time between hits of 0.5 minutes.}}\\[.1in]
\no That is we may use this value as the expected value for $Y$: $E[Y] = 0.5$.
    
\foilhead[-.75in]{\textcolor{blue}{Exponential Distribution: Example (continued...)}}\vspace{.01in}  
\no Since we know, that $E[Y]$ for an exponential RV is  $1/\lambda$, setting  $1/\lambda=0.5$ we get $ \lambda=2 $ as a 
reasonable choice for the rate parameter.\\[.1in]
\no If our model is correct, $\lambda$ describes the rate, at which this web page is hit!\\[.1in]
\no {\textcolor{magenta}{Using the above model we can answer questions like...}}\\[.1in]   
\no {\textcolor{cyan}{What is the probability that we have to wait at most 40 seconds to observe the 
    first hit?}}\\[.1in]
\no {\it (Since we are working in time units of minutes, we need to express the 40 seconds above as $2/3$ minutes.)}\\[.1in] 
\no {\textcolor{cyan}{Thus, we compute the probability that we have to wait at most 2/3 min to 
    observe the first hit:}}\\[-.2in]
    \[
    P(Y \le 2/3) =  Exp_{\lambda}(2/3) = 1 - e^{-2/3 \cdot 2} \approx 0.736
    \]
\no {\textcolor{red}{Note carefully that we just used the cdf of $Y$ (that we know) for the above calculation.}}

\foilhead[-.75in]{\textcolor{blue}{Exponential Distribution: Example (continued...)}}\vspace{.01in}  
\no {\textcolor{cyan}{How long do we have to wait at most, to observe a first hit with 
    a probability of 0.9?}\\[.1in]  
    {\it This is the reverse of what we have computed  
    so far, because here we want to find a $t$, for which $P(Y \le t) = 0.9$:}
    \begin{eqnarray*}
    && P(Y \le t) = 0.9 \\
    && \iff 1 - e^{-2t} = 0.9 \\
    && \iff e^{-2t} = 0.1 \\
    && \iff t = -0.5 \ln{0.1} \approx 1.15 \text{ (min) - that's 
    approx. 69 seconds.}
    \end{eqnarray*}
\foilhead[-.75in]{\textcolor{blue}{Memoryless property of the Exponential distribution}}\vspace{.01in}      
\no {\textcolor{magenta}{  In the \emph{Hits on a web page example} we said that we start to observe the web page 
a time point 0.}}\\[.1in]
\no   {\textcolor{cyan}{Does the choice of this time point affect our analysis in any way? }} \\[.1in]   
\no If there is no hit after the first minute after we started, what is the probability, that
 we have to wait for another $40$ seconds for the first hit?\\[.1in]
\no {\textcolor{cyan}{The probability we want to compute is a conditional probability.}}\\[.1in]
\no The probability we need is\\[.1in]
\no \h \h   $P(\text{wait for 1 min and 40 sec for the first hit}|\text{no hits during the 1st min})$\\[.1in]
\no Recall that, we used  the random variable $Y$ to denote the waiting time till the first hit.\\[.2in]
\no It is easy to see that the required probability is then $P(Y \le 5/3|Y>1)$
\foilhead[-.75in]{\textcolor{blue}{Memoryless property (continued...)}}\vspace{.01in}
\no Now, recall that under the  model we assumed, $Y \sim Exp(2.0)$
\begin{eqnarray*}
P(Y \le 5/3|Y>1) &=&\frac{P( Y \le 5/3 \cap Y > 1)}{P(Y > 1)} =\frac{P(1 < Y \le 5/3)}{1- P(Y < 1)}\\
&=&  \frac{Exp_2(5/3)-Exp_2(1)}{1-Exp_2(1)}=\frac{e^{-2} - e^{-10/3}}{e^{-2}} = 0.736.\\[.1in]
\end{eqnarray*}
\no {\textcolor{red}{This is  exactly the same probability as when we started at time 0!!!}}\\[.1in]
\no Note again that we are using the cdf of the Exponential distribution with $\lambda=2$.
\no That is, $P(X\le t)=F_{X}(t) = 1 - e^{-2 x}  \text{ if } x \ge 0 $\\[.1in]
\no The result of this example is no coincidence. We can state this as a theorem.

\foilhead[-.75in]{\textcolor{blue}{Memoryless property of the Exponential distribution}}\vspace{.1in} 
\no \hspace*{1in} $P(Y \le t + s | Y \ge s) = 1 - e^{-\lambda t} = P(Y \le t)$\\[.15in]
\no A proof of this result is given in Baron (p.84).\\[.2in] 
\no This means: a random variable with an exponential distribution 
 {\textcolor{cyan}{forgets}} about its past. This is called the {\it memoryless 
property} of the exponential distribution.\\[.1in]
\no An electrical or mechanical device whose lifetime  {\textcolor{cyan}{we model as an 
exponential variable}} therefore \emph{stays as good as new} until it 
suddenly breaks, i.e. we assume that there's no aging process.

\no Exponential is the only continuous distribution that has this property. We earlier saw that a discrete distribution (Geometric) had a similar property.


\foilhead[-.8in]{\textcolor{blue}{Gamma Distribution}}\vspace{.1in} 
\no This distribution is used to model total waiting time of a procedure that consists of 
$\alpha$ independent stages, each stage with a waiting time having a distribution $Exp_{\lambda}$.\\[.1in]
\no Then the total time has a Gamma disribution with parameters $\alpha$ and $\lambda$.\\[.1in]
\no  {\textcolor{magenta}{Gamma density}} {\textcolor{cyan}{A random variable $X$ has {\it gamma density} if}}\\[.15in]
\hspace*{1.5in} $ f(x)= \frac{\lambda^\alpha}{\Gamma(\alpha)}x^{\alpha-1}e^{-\lambda x},\qquad x>0$\\[.15in]
\no {\textcolor{cyan}{$\lambda$ is called the \emph{rate parameter} and $\alpha$ is called the \emph{shape parameter}}}\\[.1in]
\no {\textcolor{cyan}{$\Gamma(\alpha)$ is the {\textcolor{red}{Gamma function}}, an integral that is defined on p.433 (Baron)}}\\[.1in]
\no We say that the random variable $X \sim Gamma(\alpha,\lambda)$\\[.1in]
\no When $\alpha$ is an integer (this is the case with most applications we'll discuss), the gamma random variable can be represented as the sum of $\alpha \ iid\ Exp_{\lambda}$ random variables. It follows that $Gam_{1,\lambda} \equiv Exp_{\lambda}$
\foilhead[-.8in]{\textcolor{blue}{Density functions of gamma variables for different 
    shape parameters 0.5, 1, and 1.5.}}\vspace{10pt}
    \centerline{\includegraphics[scale=.7]{gamma_pdf2.pdf}}
\foilhead[-.75in]{\textcolor{blue}{Properties of the  Gamma Distribution}}\vspace{.1in} 
\no {\textcolor{red}{Mean}} and {\textcolor{red}{Variance}} are obtained using integration (see p. 85/86 of Baron). They 
are:
\vspace*{-.2in}
\begin{eqnarray*}
    E[X] &=& \frac{\alpha}{\lambda} \\
    Var[X] &=& \frac{\alpha}{\lambda^{2}}
\end{eqnarray*}
The {\textcolor{red}{cdf}},  $Gam_{\alpha,\lambda}(t) = F_{X}(t)$ is of the form
$$ F_{X}(t)=\int_0^t f(x)dx=\frac{\lambda^\alpha}{\Gamma(\alpha)}\int_0^tx^{\alpha-1}e^{-\lambda x}dx$$
The computation of the cdf is not trivial. Tabulated values of the \emph{incomplete Gamma function} is to evaluate the gamma
cdf. It can be computed for small integer values of $\alpha$ by repeated integration by parts.
\end{document}   














