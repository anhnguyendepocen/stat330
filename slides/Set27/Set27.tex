\documentclass[20pt,landscape]{foils}
\usepackage{amsmath, amssymb, amsthm}
\usepackage{color}
\usepackage{hyperref}
%\usepackage{pause}
\usepackage{graphicx}
\usepackage{epsfig}
%\usepackage{geometry}
%\geometry{headsep=3ex,hscale=0.9}
%\usepackage{graphicx}
\newcommand{\bd}{\textbf}
\newcommand{\no}{\noindent}
\newcommand{\un}{\underline}
\newcommand{\bi}{\begin{itemize}}
\newcommand{\ei}{\end{itemize}}
\newcommand{\be}{\begin{enumerate}}
\newcommand{\ee}{\end{enumerate}}
\newcommand{\bc}{\begin{center}}
\newcommand{\ec}{\end{center}}
\newcommand \h {\hspace*{.3in}}
\newcommand{\bul}{\hspace*{.1in}{\textcolor{red}{$\bullet$ \ }}}
\newcommand{\xbar}{\bar{x}}
\rightheader{Stat 330 (Fall 2016): slide set 27}


\begin{document}
\LogoOff

\foilhead[1.3in]{}
\centerline{\LARGE \textcolor{blue}{Slide set 27}}
\vspace{0.3in}
\centerline{\large Stat 330 (Fall 2016)}
\vspace{0.2in}
\centerline{\tiny Last update: \today}
\setcounter{page}{0}

\foilhead[-.8in]{\textcolor{blue}{Examples for CI of proportion}}
\no {\textcolor{magenta}{Example 1:} Suppose we want to estimate the fraction of records in the 2000 IRS data base that have a taxable income
over 35K.\\[.1in]
\no {\textcolor{magenta}{Question:} We want to get a $98\%$ confidence interval and wish to estimate the quantity to be within 0.01.  How many samples we need?\\[.1in]
\no $\spadesuit$  The {\textcolor{magenta}{size} of CI is actually $0.02$ to satisfy the desire condition  (W.L.O.G., \underline{we choose a conservative confidence interval
for easy computation}):\\[.1in]
\no $\spadesuit$ Using the second definition: recall that $P(|\hat{\theta}-\theta|<e)\geq 1-\alpha$ is the second definition, we have
$$2e\leq 0.02\iff \frac{z_{\alpha/2}}{2\sqrt{n}}\leq 0.01\iff \sqrt{n}\geq \frac{2.33}{2\cdot 0.01}=116.5$$
so that $n\geq 13573$.\\[.1in]
\no $\spadesuit$ $n\geq 0.25\left(\frac{z_{\alpha/2}}{\Delta}\right)^2$, $\Delta$ is the half of the desired size of confidence interval.

\newpage
\no {\textcolor{magenta}{Example 2:} Suppose that we are interested in the large time probability $p$ that a
server is available. Doing 100 simulations has shown, that in 60 of them a server was available at time $t = 1000$ hrs.
What is a $95\%$  confidence interval for this probability?\\[.15in]
\no $\spadesuit$ If 60 out of 100 simulations showed a free server, we can use $\hat{p} = 60/100 = 0.6$ as an estimate for $p$, or, use the conservative one $\hat{p}=0.5$.\\[.1in]
\no $\spadesuit$ For a $95\%$ confidence interval, $z_{\alpha/2}=z_{(1-0.95)/2} = \Phi^{-1}(0.975) = 1.96$.
The {\textcolor{magenta}{conservative confidence interval} is:
$$    \hat{p} \pm z_{\alpha/2} \frac{1}{2 \sqrt{n}} = 0.6 \pm 1.96 \frac{1}{2
    \cdot \sqrt{100}} = 0.6 \pm 0.098.
  $$
\no For the confidence interval using {\textcolor{magenta}{substitution} we get:
$$
    \hat{p} \pm z_{\alpha/2} \sqrt{\frac{\hat{p} (1- \hat{p})}{n}} = 0.6 \pm 1.96
    \sqrt{\frac{0.6 \cdot 0.4}{100}} =
     0.6 \pm 0.096.
   $$

\foilhead[-.8in]{\textcolor{blue}{Two Populations}}
\no {\textcolor{magenta}{CI for mean difference $\mu_1-\mu_2$, or, the difference of two proportions, $p_1-p_2$}}  \\[.1in]
\no $\spadesuit$ $\bar{X}_1-\bar{X}_2$ and $\hat{p}_1-\hat{p}_2$ are the unbiased estimators for $\mu_1-\mu_2$ and $p_1-p_2$, respectively.\\[.1in]
\no $\spadesuit$ Confidence intervals are summarized below:\\[.15in]
\begin{tabular}{p{4in}|p{5in}}
    \textcolor{red}{large $n$} confidence interval for $\mu_{1}- \mu_{2}$ (based on
    independent $\bar{X}_{1}$ and  $\bar{X}_{2}$) &     \textcolor{red}{large $n$} confidence
    interval for $p_{1}- p_{2}$ (based on
    independent $\hat{p}_{1}$ and  $\hat{p}_{2}$) \\\hline
    & \\
    $\bar{x}_{1} - \bar{x}_{2} \pm z_{\alpha/2} \sqrt{\frac{\sigma_{1}^{2}}{n_{1}} +
    \frac{\sigma_{2}^{2}}{n_{2}}}$ 
   & $\hat{p}_{1}-\hat{p}_{2} \pm  \frac{z_{\alpha/2}}{2}
    \sqrt{\frac{1}{n_{1}}+\frac{1}{n_{2}}}$ (\textcolor{magenta}{conservative}) \\ 
    ({\textcolor{magenta}{when $\sigma_{1}^{2},\sigma_{2}^{2}$ unknown, substitute $s_{1}^{2}, s_{2}^{2}$ respectively }})    & or \\
    &  $\hat{p}_{1}-\hat{p}_{2} \pm z_{\alpha/2}
    \sqrt{\frac{\hat{p}_{1}(1-\hat{p}_{1})}{n_{1}} + \frac{\hat{p}_{2}{(1-\hat{p}_{2})}}{n_{2}}}$ (\textcolor{magenta}{substitution})
\end{tabular}

%\foilhead[-.8in]{\textcolor{blue}{Two Populations: Simple derivation}}
%\no {\textcolor{magenta}{Derivation:} The arguments are very similar in both cases - we will only discuss the confidence interval for the difference between the two means.
%
%\begin{enumerate}
%\item $\bar{X}_{1} - \bar{X}_{2}$ is approximately normal, since
%$\bar{X}_{1}$ and $\bar{X}_{2}$ are approximately normal, with
%($\bar{X}_{1}$, $\bar{X}_{2}$ are independent)
%
%\item $\bar{X}_i\sim N(\mu_i, \sigma_i^2/n_i)$ for $i=1,2$
%    \begin{eqnarray*}
%    E[\bar{X}_{1} - \bar{X}_{2}] &=& E[\bar{X}_{1}] -
%    E[\bar{X}_{2}] = \mu_{1} - \mu_{2} \\
%    Var[\bar{X}_{1} - \bar{X}_{2}] &=& Var[\bar{X}_{1}] + (-1)^{2}
%    Var[\bar{X}_{2}] = \frac{\sigma^{2}_1}{n_{1}} + \frac{\sigma^{2}_2}{n_{2}}
%\end{eqnarray*}
%
%\item Then we can use the similar arguments as before and get a C.I. for
%$\mu_{1} - \mu_{2}$ as shown above.
%\end{enumerate}

\foilhead[-.8in]{\textcolor{blue}{Two Populations: Example}}\vspace{.5cm}
\no {\textcolor{magenta}{Example 1:} Assume, we have two parts of the IRS database: East Coast and West Coast. We want to compare the mean taxable income between reported from
    the two regions in 2000.\\[.1in]
\no
\begin{tabular}{rcc}
	& East Coast & West Coast \\
	\# of sampled records: & $n_{1} = 1000$ & $n_{2} = 2000$ \\
	mean taxable income: & $\bar{x}_{1} = \$ 37200$ & $\bar{x}_{2} = \$ 42000$
	\\
	standard deviation: & $s_{1} = \$ 10100$ & $s_{2} = \$ 15600$ \\
    \end{tabular}  \\[.1in]
\no $\spadesuit$  We can, for example, compute a 2 sided $95\%$ confidence interval for $\mu_{1} - \mu_{2}$ = difference in mean taxable income as reported from 2000 tax return between East and West Coast as following: \\[.1in]
\no $\spadesuit$ \[
    37000 - 42000 \pm z_{\alpha/2}\sqrt{\frac{10100^{2}}{1000} +
    \frac{15600^{2}}{2000}} =
    -5000 \pm 927
    \]
\no $\spadesuit$ Note: this shows pretty conclusively that the mean West Coast taxable income is higher than the mean East Coast taxable income (in the report from 2000). The interval contains only negative numbers \\[.1in]
\no $\spadesuit$ However, if it contained the 0, the message wouldn't be so clear.


\no  {\textcolor{magenta}{Example 2:} Two different digital communication systems send 100 large messages
via each system and determine how many are corrupted in transmission, $\hat{p}_{1} = 0.05$ and $\hat{p_{2}} = 0.10$. What's the difference in the corruption rates? Find a $98\%$
confidence interval.
\\[.1in]
\no $\heartsuit$
Use:
\[
0.05 - 0.1 \pm 2.33 \cdot \sqrt{\frac{0.05 \cdot 0.95}{100} + \frac{0.10 \cdot
0.90}{100}} = -0.05 \pm 0.086
\]
\no $\heartsuit$
This calculation tells us, that based on these sample sizes, we don't
even have a solid idea about the sign of $p_{1} - p_{2}$, i.e. we
can't tell which of the two $p_{i}$ is larger. 

\foilhead[-.8in]{\textcolor{blue}{Small samples when the standard deviation $\sigma$ is unknown}}\vspace{.5cm}
\no  {\textcolor{magenta}{Single Population:}\\[.2in]
\no \bul If the standard deviation $\sigma$ is unknown, but sample $X_{1},...,X_{n}$
can be assumed to come from a \no  {\textcolor{magenta}{Normal distribution}}, then instead of using $z_{\alpha/2}$,
we may use $t_{n-1,\,\alpha/2}$ which is the corresponding percentile of the $t$ distribution with $n-1$ degrees of freedom\\[.2in]
\no \bul The resulting $100\times(1-\alpha)\%$ confidence interval for $\mu$ is 
\[\bar{x}\pm t_{n-1,\,\alpha/2} \cdot \frac{s}{\sqrt{n}}.\]
This is helpful when sample size $n$ is small, since the CLT does not apply.\\[.2in]
\no \bul {\textcolor{magenta}{See pages below for a note on the t-distribution.}
\foilhead[-.8in]{\textcolor{blue}{Small samples; $\sigma$ is unknown: continued...}}\vspace{.5cm}
\no  {\textcolor{magenta}{Two Populations:}\\[.2in]
\no \bul Population variances $\sigma_1^{2}$ and $\sigma_2^{2}$ are {\textcolor{magenta}{unknown}}.
\begin{enumerate}
\item[\bul] {\textcolor{magenta}{assume equal variances }} $\sigma_1^{2}=\sigma_2^{2}=\sigma^{2}$\\[.2in]
The $100\times(1-\alpha)\%$ confidence interval for $\theta=\mu_1-\mu_2$
is \[
\bar{x}_1-\bar{x}_2 \pm t_{n+m-2,\alpha/2} \cdot s_{p}\sqrt{\frac{1}{n}+\frac{1}{m}},\]
where $s_{p}^{2}$ is the {\textcolor{red}{pooled variance}} calculated as \[
s_{p}^{2}=\frac{(n-1)s_1^{2}+(m-1)s_2^{2}}{n+m-2}\]
\item[\bul] {\textcolor{magenta}{assume unequal variances}} $\sigma_1^{2}\neq\sigma_2^{2}$\\[.2in]
The $100\times(1-\alpha)\%$ confidence interval for $\theta=\sigma_1-\sigma_2$
is\[
 \bar{x}_1-\bar{x}_2\pm t_{\nu,\alpha/2}\sqrt{\frac{s_1^{2}}{n}+\frac{s_2^{2}}{m}},\]
where \[
\nu=\frac{\left(\frac{s_1^{2}}{n}+\frac{s_2^{2}}{m}\right)^{2}}{\frac{s_1^{4}}{n^{2}(n-1)}+\frac{s_2^{4}}{m^{2}(m-1)}} \quad\quad \mbox{(rounded to the nearest integer)}\]

\end{enumerate}
\foilhead[-.8in]{\textcolor{blue}{t distribution}}
\no\bul A random variable $T$ that is of form \[
T=\frac{Z}{\sqrt{V/\nu}}\]
is said to have a {\textcolor{magenta}{t distribution}} with $\nu$ degress of freedom
for random variables $Z$ and $V$ such that:
\h \h \begin{enumerate}
\item $Z\sim N(0,1)$ and $V\sim\chi_{\nu}^{2}$ (a {\textcolor{magenta}{Chi-square distribution}} with $\nu$ degrees of freedom).
\item $Z$ and $V$ are independent.
\item Chi-square distribution is a special case of the Gamma distribution.
\end{enumerate}
\newpage
\no \bul The diagram shows the probability density function of this distribution for several different degrees of freedom:
\begin{figure}[h]
  \centering
  \epsfig{file=Student_t_pdf.pdf, height=7cm,width=12cm}
\end{figure}
\no \bul Observe that as the degrees of freedom increases the shape of the t-ditribution tends towards that of the  Normal distribution.\\[.1in]
\no \bul Read percentiles of the t distribution from Table A5 of Baron's textbook.



\end{document}




