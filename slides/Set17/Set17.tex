\documentclass[20pt,landscape]{foils}
\usepackage{amsmath, amssymb, amsthm}
\usepackage{color}
\usepackage{hyperref}
%\usepackage{pause}
\usepackage{graphicx}
\usepackage{epsfig}
%\usepackage{geometry}
%\geometry{headsep=3ex,hscale=0.9}
\newcommand{\bd}{\textbf}
\newcommand{\no}{\noindent}
\newcommand{\un}{\underline}
\newcommand{\bi}{\begin{itemize}}
\newcommand{\ei}{\end{itemize}}
\newcommand{\be}{\begin{enumerate}}
\newcommand{\ee}{\end{enumerate}}
\newcommand{\bc}{\begin{center}}
\newcommand{\ec}{\end{center}}
\newcommand \h {\hspace*{.3in}}
\newcommand{\bul}{\hspace*{.3in}{\textcolor{red}{$\bullet$ \ }}}
\newcommand{\xbar}{\bar{x}}
\rightheader{Stat 330 (Fall 2016): slide set 17}

\begin{document}
\LogoOff

\foilhead[1.3in]{}
\centerline{\LARGE \textcolor{blue}{Slide set 17}}
\vspace{0.3in}
\centerline{\large Stat 330 (Fall 2016)}
\vspace{0.2in}
\centerline{\tiny Last update: \today}
\setcounter{page}{0}


\foilhead[-.8in]{\textcolor{blue}{Example 6.9 (Baron)}}
\no {\textcolor{magenta}{Shared Device}}. A computer is shared by 2 users who send
tasks to a computer remotely and work independently. At any minute,
any connected user may disconnect with probability 0.5, and any disconnected
user may connect with a new task with probability 0.2. Let $X(t)$
be the number of concurrent users at time $t$ (minutes). This is
a Markov chain with 3 states: 0, 1, and 2. 
\begin{enumerate}
\item {\textcolor{cyan}{Compute the transition probabilities.}}\\[.1in]
First assume $X(0)=0$ i.e. no users connected at time $t=0$.\\[.1in]
The $X(1)$, number that connects within the next minute, is $\sim Bin(2,.2)$\\[.1in]
Thus we can calculate 
$$p_{00}= (.8)^2=.64,\ p_{01}= 2(.2)(.8)=.32,\  p_{02}= (.2)^2(.8)^0=.04$$
Next, assume $X(0)=1$, i.e. one user connected at time $t=1$.\\[.1in]
The number that connects within the next minute, is $\sim Bin(1,.2)$, and\\
 the number that disconnects within the next minute, is $\sim Bin(1,.5)$.\\[.1in]
Thus, the state can change from $1 \rightarrow 0,\,1 \rightarrow 1,$ or from $1 \rightarrow 2$.\\
Calculating the probabilities for those transitions:\\[.1in]
\bul $p_{10}= $ P(one disconnects AND zero connects)$=(.5) \cdot (.8)=.40$\\[.1in]
\bul $p_{11}= $ P((one disconnects AND one connects) OR \\
\hspace*{3in}(zero disconnects AND zero connects))\\
\hspace*{3in} $=(.5)\cdot (.2) + (.8)\cdot (.5)=.50 $\\[.1in]
\bul $p_{12}= $ P(zero disconnects AND one connects) $=(.5)\cdot (.2) =.10 $\\[.1in]
Finally, when $X(0)=2$ i.e. no users can  connect at time $t=1$, but the number that disconnects within the next minute, is $\sim Bin(2,.5)$.
Thus \\[.1in]
\bul $p_{20}= (.5)^2(.5)^0=.25,\\
\hspace*{2in} \,p_{21}= 2(.5)^1(.5)^1=.50,\,p_{22}= (.5)^0(.5)^2=.25$ 

Hence the transition probability matrix is\\[-.2in]
\[P=\begin{pmatrix}.64 & .32 & .04 \\
.40 & .50 & .10\\
.25 & .50 & .25
\end{pmatrix}\]
The 2-step transition probability matrix is\\[-.2in]
\[P^{(2)}=P^2=
  \begin{pmatrix}.64 & .32 & .04 \\
.40 & .50 & .10\\
.25 & .50 & .25
\end{pmatrix} \cdot 
\begin{pmatrix}.64 & .32 & .04 \\
.40 & .50 & .10\\
.25 & .50 & .25
\end{pmatrix} =
\begin{pmatrix}.5476  & .3848 & .0676 \\
.4810 & .4280 & .0910\\
.4225 & .4550 & .1225
\end{pmatrix}
\]
So if both user's are connected at 10:00, then the $P(\text{''no users connected at 10:02''})=.4225$ and\\ 
$P(\text{''one user connected at 10:02''})=.4550$ etc.\\[-.5in]

\item {\textcolor{cyan}{If we know that there are 2 users connected at 10:00, that is, the
initial distribution is $P_{0}=(0,\,0,\,1)$. Then what is the distribution
of the number of users at 10:02, i.e. $P_{2}$? What if the initial
distribution is $P_{0}=(1/3,\,1/3,\,1/3)$?}}\\
Use $ P_2=P_0 \cdot P_{(2)}$ to get 
$$ P_2= (0,\,0,\,1) \cdot
\begin{pmatrix}.5476  & .3848 & .0676 \\
.4810 & .4280 & .0910\\
.4225 & .4550 & .1225
\end{pmatrix}
= (.4225,\, .4550,\,.1225)$$
and,
$$ P_2= (1/3,\,1/3,\,1/3) \cdot
\begin{pmatrix}.5476  & .3848 & .0676 \\
.4810 & .4280 & .0910\\
.4225 & .4550 & .1225
\end{pmatrix}
= ( .4837,\, .4226,\, .0937)$$
\item {\textcolor{cyan}{In practice, we may be interested in $P_{h}$ after many transitions,
i.e. for large $h$. How do we calculate it, say the distribution
of states at 11:00 (60-step transition, i.e.,$h=60$).}}
\end{enumerate}

\foilhead[-.8in]{\textcolor{blue}{Steady-state distribution}}
\no For a Markov chain $X(t)$, if the limiting distribution of $P_{h}$, that is,
the following probabilities exist, \\[.1in]
\hspace*{2in} $\pi_{x}=\lim_{h\to\infty}P_{h}(x),\ \ x\in\mathcal{X},$\\[.1in]
then \textbf{$\pi$ }(not the number $3.1415926\cdots$) is called
a \textcolor{magenta}{steady-state distribution} of Markov chain $X(t)$. 
\begin{enumerate}
\item {\textcolor{cyan}{How to compute the steady-state distribution?}} \\
$\pi$, i.e. $(\pi_{1},\pi_{2},\ldots,\pi_{n})$ is the solution
to the following set of linear equations 
$$\pi P=\pi,\qquad \sum_{x}\pi_{x}=1.$$\\
Note that $\pi P$ is matrix multiplication, $(\pi_{1},\pi_{2},\ldots,\pi_{n}) \cdot P$,
that is  an $n\times n$ matrix $P$ multiplied by a $1\times n$ matrix (row vector). 
It is easy to see that $\sum_{x}\pi_{x}=1$ as the end-up
states should be in $\mathcal{X}$.
 
\item {\textcolor{cyan}{What is meant by the ''steady state'' of a Markov chain?}}\\
Suppose the system has reached its steady state, so that the distribution
of the states is $P_{t}=\pi$. Then the state distribution after one
more transition is $P_{t+1}=P_{t}P=\pi P=\pi=P_{t}$. This means that,
given a chain is in a steady state, a transition \emph{does not affect} the state distribution.
In this sense, it is steady.\\
\vspace*{-.5in}

\item {\textcolor{cyan}{Do we have a limit for $P^{h}$, the $h$-step transition matrix?}}\\
Yes, the limit is\\
\vspace*{-.7in}

\[\Pi=\lim_{h\to\infty}P^h=\begin{pmatrix}\pi_1 & \pi_2 & \cdots &\pi_n\\
\pi_1 & \pi_2 & \cdots & \pi_n\\
\vdots & \vdots & \ddots & \vdots\\
\pi_1 & \pi_2 & \cdots & \pi_n
\end{pmatrix}.\]
\vspace*{-.4in}

\item {\textcolor{cyan}{Does the steady-state distribution always exist?}}\\
No.
\end{enumerate}

\no \foilhead[-.8in]{\textcolor{blue}{Examples:}}
\no \textcolor{magenta}{Ames weather problem:}\\
 \[P=\begin{pmatrix}0.7 & 0.3\\
0.4 &  0.6
\end{pmatrix},\] then
\[P^2=\begin{pmatrix}0.61 & 0.39\\
0.52 & 0.48
\end{pmatrix}, \, P^3=\begin{pmatrix}0.583 & 0.417\\
0.556 & 0.444
\end{pmatrix},P^{14}=\begin{pmatrix}0.5714286 & 0.4285714\\
0.5714285 & 0.4285715\
\end{pmatrix},\]\[P^{15}\approx \ldots\approx P^{30}\approx\begin{pmatrix} 0.5714286 & 0.4285714 \\
0.5714286 & 0.4285714
\end{pmatrix}=\begin{pmatrix}4/7 & 3/7\\
4/7&3/7\end{pmatrix}.\]
For any given starting state distribution $P_{0}=(P_{0}(0),\,P_{0}(1))=(p,\,1-p)$,\\
we see that
 \[(p,\,1-p) \begin{pmatrix}4/7 & 3/7\\
4/7&3/7\end{pmatrix}=(4/7, 3/7).\]
\textcolor{magenta}{Using the theory to obtain this steady-state distribution, solve the equation:}\\[-.25in]
$$ (\pi_1,\,\pi_2) \cdot \begin{pmatrix}0.7 & 0.3 \\
0.4 & 0.6
\end{pmatrix}=(\pi_1,\,\pi_2)$$
Need to solve the system of equations:\\[-.3in]
\begin{eqnarray*}
0.7 \pi_1  +0.4 \pi_2 & = & \pi_1\\
0.3 \pi_1  +0.6 \pi_2 & = & \pi_2
\end{eqnarray*}\\[-.3in]
giving the solution $(\pi_1,\,\pi_2)=(4/7,\,3/7)$\\[.1in]
\no The story is: No matter what assumptions you make about the initial
probability distribution, after a large number of steps have been
taken the probability distribution is approximately $(4/7,\,3/7)$.
This $(4/7,\,3/7)$ is still a distribution ($4/7+3/7=1$), which we
would call a \textcolor{magenta}{steady-state }distribution.\\[.1in]
\no (\textcolor{magenta}{Shared device:} (See Baron Example 6.12).

%\no \foilhead[-.8in]{\textcolor{blue}{Regular Markov Chain}}
%\no A Markov Chain $X(t)$ with transition matrix $P$ is said to
%be \textcolor{magenta}{regular, }if, for some $n$, \textcolor{magenta}{all} entries of $P^{n}$
%are positive.\\
%\no Any regular Markov chain has a steady-state distribution.
%\vspace*{-0.4cm}
%\begin{enumerate}
%\item {\textcolor{cyan}{Not every Markov chain has a steady-state distribution. Why?}}\\
%Consider the following transition matrix:\\[-.2in]
% \[P=\begin{pmatrix}0&1\\1&0\end{pmatrix},\] then \[P^{2k}=\begin{pmatrix}1&0\\0&1\end{pmatrix},\ \  P^{2k-1}=\begin{pmatrix}0&1\\1&0\end{pmatrix},\  \ \forall k\in\mathbb N.\]
%\item {\textcolor{cyan}{As long as we find \textcolor{magenta}{some} $n$ such that \textcolor{magenta}{all} entries
%of $P^{n}$ are positive, then the chain is regular. This does not mean that
%a regular Markov chain has to possess this property for all $n$.}}\\
%
%\no  Consider the following transition matrix,\\[-.2in]
%\[P=\begin{pmatrix}0&0&1\\2/3&0&1/3\\1/2&1/4&1/4\end{pmatrix},\]then
%\[
%  P^2=\begin{pmatrix}.500 & .250 & .250\\
%    .167 & .083 & .750\\
%    .292& .063& .646\end{pmatrix}
%  %P^2=\begin{pmatrix}2/3&0&1/3\\1/6&3/4&1/12&\\7/24&9/16&7/48\end{pmatrix}
%  %,\, P^3=\begin{pmatrix}1/6&3/4&1/12\\13/24&3/16&13/48\\43/96&21/64&43/192\end{pmatrix}.
%\]
%\no This Markov chain is regular, though elements of some transition matrices are non-positive!.
%\end{enumerate}





\end{document}




