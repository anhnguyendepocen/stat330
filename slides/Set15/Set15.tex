\documentclass[20pt,landscape]{foils}
\usepackage{amsmath, amssymb, amsthm}
\usepackage{amstext}
\usepackage{amsgen}
\usepackage{amsxtra}
\usepackage{amsgen}
\usepackage{amsthm}
\usepackage{color}
\usepackage{hyperref}
%\usepackage{pause}
\usepackage{graphicx}
\usepackage{epsfig}
%\usepackage{geometry}
%\geometry{headsep=3ex,hscale=0.9}
\newcommand{\bd}{\textbf}
\newcommand{\no}{\noindent}
\newcommand{\un}{\underline}
\newcommand{\bi}{\begin{itemize}}
\newcommand{\ei}{\end{itemize}}
\newcommand{\be}{\begin{enumerate}}
\newcommand{\ee}{\end{enumerate}}
\newcommand{\bc}{\begin{center}}
\newcommand{\ec}{\end{center}}
\newcommand \h {\hspace*{.3in}}
\newcommand{\bul}{\hspace*{.3in}{\textcolor{red}{$\bullet$ \ }}}
\newcommand{\xbar}{\bar{x}}
\rightheader{Stat 330 (Fall 2016): slide set 15}

\begin{document}
\LogoOff

\foilhead[1.3in]{}
\centerline{\LARGE \textcolor{blue}{Slide set 15}}
\vspace{0.3in}
\centerline{\large Stat 330 (Fall 2016)}
\vspace{0.2in}
\centerline{\tiny Last update: \today}
\setcounter{page}{0}

\foilhead[-.75in]{\textcolor{blue}{ Central Limit Theorem (CLT)}}
\no {\textcolor{magenta}{Main Idea:}} {\textcolor{cyan}{Sums and averages of random variables from arbitrary distributions have approximate normal distributions for sufficiently large sample sizes.}}\\[.1in]
\no Suppose $X_1,X_2,\ldots,X_n$ are iid random variables with 
$$ E[X_i]=\mu \hspace*{.5in} Var[X_i]=\sigma^2, \qquad i=1,\ldots,n$$ 
\no  Define 
$$ \mbox{Sample Average:}\qquad \overline{X}_n=\frac{(X_1+X_2+\dots+X_n)}{n}$$
$$\mbox{Sample Sum:}\qquad S_n=X_1+X_2+\dots+X_n$$
\no For large $n$
$$ \overline{X}_n \dot{\sim} N(\mu,\sigma^2/n)$$
$$ S_n \dot{\sim} N(n\mu,n\sigma^2)$$

\foilhead[-.75in]{\textcolor{blue}{ Central Limit Theorem (CLT) (cont'd)}}
\no {\textcolor{magenta}{Use of CLT: }}  Calculate probabilities associated with averages or sums of iid random variable. these approximate distributional statements. For e.g.,\\[.1in]
\no  \hspace*{.5in} $P(a<\overline{X}_n<b)  \approx P(a<X<b)$= $\Phi\left(\frac{b-\mu}{\sigma/\sqrt{n}}\right)-\Phi\left(\frac{a-\mu}{\sigma/\sqrt{n}}\right)$\\[.1in] 
\no Recall: $\Phi$ is the cdf of the standard normal distribution i.e., $Z \sim N(0,\,1)$\\[.1in]
\no {\textcolor{magenta}{Some Example Applications of the Central Limit Theorem}}\\[.1in]
\no {\textcolor{magenta}{Example 1:}} {\textcolor{cyan}{The time I spend waiting for the bus in a day has a Uniform distribution between 2 minutes and 5 minutes.}}\\[.1in]
\no (a) {\textcolor{cyan} {How much time do I expect to spent waiting for the bus in one month (30 days)?}}\\[.1in]
\no  \hspace*{1in} Let $X_i=$ time I wait for the bus on day $i$. \\[.1in]
\no \hspace*{.5in} Then $X_1,X_2,\ldots X_{30} \sim iid\ U(2,5)$ and it follows that\\[.1in]
\no  $E[X_i]=(2+5)/2=3.5\,min,\ Var(X_i)=(5-2)^2/12=9/12=.75\,min.^2$

\foilhead[-.75in]{\textcolor{blue}{ CLT Examples (cont'd)}}
\no {\textcolor{magenta} {Let $T\equiv$ random variable representing the total waiting time for a month.}}\\[.1in] 
\no $T$ is the sum of 30 iid random variables: $T=X_1+X_2+\ldots +X_{30} $\\[.1in]
\no $E[T]= \sum_{i=1}^{30} X_i=30\mu$ where $\mu=E[X_i]=3.5$.\\[.1in] 
Thus the expected waiting time for a month $E[T]=30\times 3.5=105 \,min.$\\[.15in]
\no (b) {\textcolor{cyan} {Approximately, find the probability that I spend more than 2 hours waiting for a bus in a month.}}\\[.1in]
\no From the CLT, we have $T\dot{\sim} N(30\times 3.5, 30\times 0.75) \quad \mbox{i.e. } {\textcolor{magenta} {T\dot{\sim} N(105, 22.5)}}$\\[.1in]
\no We need the probability that $T$ is greater than 120 minutes, i.e., $P(T>120)$
\begin{eqnarray*}
P(T>120) & = & 1 -P(T \le 120)\\
 &\approx & 1-P(Z \le \frac{(120-105)}{\sqrt{22.5}})\qquad \mbox{by CLT}\\
 &=& 1- \Phi(3.16)=1-.9992112=.00079
\end{eqnarray*}
\foilhead[-.75in]{\textcolor{blue}{ CLT Example from Baron}}
\no  {\textcolor{magenta} {Example 4.13 (Allocation of Disk Space)}} {\textcolor{cyan} { A disk has free space of 330 megabytes. Is it likely to be sufficient for 300 independent images, if each image has expected size of 1 megabyte with a standard deviation of 0.5 megabytes?}}\\[.1in]
\no We have $n=300, \mu=1$ and $\sigma=0.5$. The number of images $n$ is large, so the CLT applies. Then
\begin{eqnarray*}
P(\text{sufficient space})&=& P(S_n\le 330))\\
 & = & P\left(\frac{S_n-n\mu}{\sigma \sqrt{n}}\le \frac {330-(300)(1)}{0.5\sqrt{300}})\right)\\
 &\approx &\Phi(3.46)=.9997
\end{eqnarray*} 
Since this probability is very high, the available disk space is very likely to be sufficient.
\foilhead[-.75in]{\textcolor{blue}{ CLT Big Example}}
\no  {\textcolor{cyan} {An astronomer wants to measure the distance, $d$, from the observatory to a star. Due to the variation of atmospheric conditions and imperfections in the measurement method, a single measurement will not produce the exact distance $d$. The astronomer takes $n$ measurements of the distance and uses  the sample average  to estimate the true distance. From past records of these measurements the astronomer knows the variance of a single measurement is $4\ \mbox{parsec}^2$. How many measurement should the astronomer make so that the chance that his estimate differs by $d$ by more than .5 parsecs is at most .05?}}\\[.1in]
\no \hspace*{.5in} Let $X_i$ be the $i^{th}$ measurement. The astronomer assumes that \\[.1in]
\no \hspace*{1in} $ X_1,X_2,\ldots X_n \sim iid \mbox{ with } E[X_i]=d \mbox{ and } Var[X_i]=4$\\[.1in]
\no \hspace*{2in}The estimate of $d$ is $\overline{X}_n=\frac{(X_1+X_2+\dots+X_n)}{n}$\\[.1in]
\no \hspace*{.5in} We want to find the number of measurements $n$ so that\\[.1in]
\no  \hspace*{3in} $ P(|\overline{X}_n-d|>.5)\le .05$

\foilhead[-.75in]{\textcolor{blue}{ CLT Big Example (cont'd)}}
\no We know that \\[.1in]
\no \hspace*{1in} $P(|\overline{X}_n-d|>.5) =P(\overline{X}_n-d>.5)+P(\overline{X}_n-d<-.5)$\\[.1in]
\no We use the CLT to approximate each of the probabilities on the right. From the CLT we have that\\[.1in]
\no \hspace*{3in} $ \overline{X}_n \dot{\sim} N(d,4/n)$\\[.1in]
Thus 
\begin{eqnarray*}
 P(|\overline{X}_n-d|>.5) &= &P(\overline{X}_n-d>.5)+P(\overline{X}_n-d<-.5)\\
 &=& P\left(\frac{\overline{X}_n-d}{\sqrt{4/n}}>\frac{.5}{\sqrt{4/n}}\right)+
 P\left(\frac{\overline{X}_n-d}{\sqrt{4/n}}<\frac{-.5}{\sqrt{4/n}}\right)\\
 & \approx & P\left(Z>\frac{.5}{\sqrt{4/n}}\right)+P\left(Z<\frac{-.5}{\sqrt{4/n}}\right)
\end{eqnarray*} 
\foilhead[-.75in]{\textcolor{blue}{ CLT Big Example (cont'd)}}
\begin{eqnarray*} 
 &=& 1- \Phi(\sqrt{n}/4)+\Phi(-\sqrt{n}/4)\\
 &=& 2(1-\Phi(\sqrt{n}/4))
\end{eqnarray*}
\no \bul We need to find an integer $n$ so that $2(1-\Phi(\sqrt{n}/4))$ is just less than or equal to $.05$.\\[.1in]
\no \bul We will set $2(1-\Phi(\sqrt{n^*}/4))=.05$, solve for $n^*$ and take the required number of 
measurements to be the $\lceil n^* \rceil$.\\[.1in]
\no \bul Observe that $2(1-\Phi(\sqrt{n}/4))=.05$ implies that $\Phi(\sqrt{n}/4))=.975$.\\[.1in] 
\no \bul Using the Normal cdf tables, this gives $\sqrt{n}/4=1.96$; thus $n^*=61.47$. \\[.1in]
\no \bul Thus the astronomer must take at least 62 measurements to have the accuracy specified above.

\foilhead[-.8in]{\textcolor{blue}{Normal approximation to the Binomial}}
\no  {\textcolor{magenta} {For large $n$, the binomial distribution $B_{n, p}$ is 
	approximately normal $N_{np, np(1-p)}$. Why?}}\\[.1in]	
\no Let $Y$ be a variable with a $B_{n,p}$ distribution. We know, that $Y$ is the number of successes in $n$ independent Bernoulli experiments with $P(\text{success})=p$.\\[.1in]
\no Write $Y$ as the sum of $n\ iid$ Bernoulli variables each with $\mu=E(X_i)=p$ and $\sigma^2=Var(X_i)=p(1-p)$: $ Y= X_{1} + X_{2} + \ldots + X_{n}$\\[.1in]
\no  {\textcolor{cyan} {Applying the CLT result for $S_n$, we have that  $ Y \dot{\sim} N(n \mu ,n \sigma^2)$ where $\mu=p$ and $\sigma^2=p(1-p)$. That is, $ Y \dot{\sim} N(np ,np(1-p)).$}}\\[.1in]
\no Use this approximation only when $np$	and $n(1-p)$ are both $>5$; the approximation is  pretty good when $np$	and $n(1-p)$ are both $>20$.\\[.1in]
\no When either of  $np$	or $n(1-p)$ are $<20$, a {\textcolor{magenta} {continuity correction} is needed (see Baron p.94).
\end{document}   





  









