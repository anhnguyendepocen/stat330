\documentclass[20pt,landscape]{foils}
\usepackage{amsmath, amssymb, amsthm}
\usepackage{color}
\usepackage{hyperref}
%\usepackage{pause}
\usepackage{graphicx}
\usepackage{epsfig}
%\usepackage{geometry}
%\geometry{headsep=3ex,hscale=0.9}
\newcommand{\bd}{\textbf}
\newcommand{\no}{\noindent}
\newcommand{\un}{\underline}
\newcommand{\bi}{\begin{itemize}}
\newcommand{\ei}{\end{itemize}}
\newcommand{\be}{\begin{enumerate}}
\newcommand{\ee}{\end{enumerate}}
\newcommand{\bc}{\begin{center}}
\newcommand{\ec}{\end{center}}
\newcommand \h {\hspace*{.3in}}
\newcommand{\bul}{\hspace*{.1in}{\textcolor{red}{$\bullet$ \ }}}
\newcommand{\xbar}{\bar{x}}
\rightheader{Stat 330 (Fall 2015): slide set 23}

\begin{document}
\LogoOff

\foilhead[1.3in]{}
\centerline{\LARGE \textcolor{blue}{Slide set 23}}
\vspace{0.3in}
\centerline{\large Stat 330 (Fall 2015)}
\vspace{0.2in}
\centerline{\tiny Last update: \today}
\setcounter{page}{0}


\foilhead[-.8in]{\textcolor{blue}{Statistical inference}}
\no {\textcolor{blue}{Question:} What is Statistics?\\[.1in]
\no  {\textcolor{magenta}{Statistics:} is the science and art of studying data. It involves collecting, classifying, summarizing, organizing, analyzing, and interpreting numerical information.  \\[.1in]
\no {\textcolor{magenta}{More:}} In general, there are two different processes in statistics: \\[.1in]
\h  {\textcolor{red}{(1)}{ describing sets of data } \\[.1in]
\h {\textcolor{red}{(2)}{ drawing conclusions about the sets of data based on sampling}\\[.1in]
\no The application of statistics can be divided into two broad areas: {\textcolor{magenta}{descriptive statistics}} and {\textcolor{magenta}{inferential statistics}}.\\[.1in]
\no {\textcolor{magenta}{Main areas in statistical inference are:}} estimation of parameters, evaluation of plausibility of values (hypothesis testing), and prediction of future values


\foilhead[-.8in]{\textcolor{blue}{An example}}

\no  {\textcolor{magenta}{The share of the credit card market by issuers:} this is a descriptive summary of some numerical information \\[.1in]
\no  \begin{figure}[h]
  \centering
  \epsfig{file=f1.pdf, height=9cm}

\end{figure}

\foilhead[-.8in]{\textcolor{blue}{Fundamental elements of statistics (a)}}
\no  {\textcolor{magenta}{Population and sample:}  \\[.1in]
\no $\heartsuit$ {\textcolor{red}{Population}}: all possible measurements or outcomes that are of interest to us in a particular study \\[.1in]
\no $\heartsuit$ {\textcolor{red}{Sample}}: a portion of the population that is representative of the population from which it was selected \\[.1in]
\no $\spadesuit$ What is the difference?  \\[.1in]
\no Thinking about an example where we want to know the average height of all people in the world, how can we do that? Can we do that?   \\[.1in]
\no  $\spadesuit$  A {\textcolor{magenta}{population}} basically has all units of interest; while all of them cannot be observed, a selected subset can be observed and forms the {\textcolor{magenta}{sample}}. \\[.1in]
\no $\clubsuit$ Sample is used to make statement(s) about the whole population.

\foilhead[-.8in]{\textcolor{blue}{Fundamental elements of statistics (b)}}\vspace{.1cm}
\no  {\textcolor{magenta}{Statistical inference:} is the process of drawing conclusions from data that are subject to random variation, for instance, observational errors or sampling variation. \\[.1in]
\no {\textcolor{magenta}{Example:} A large paint retailer has had numerous complaints from customers about underfilled paint cans. As a result, the retailer has begun inspecting incoming shipments of paint from the supplier. A recent shipment contained 2440 gallon-sized cans. The retailer randomly samples 100 cans and weighed each on a scale capable of measuring weight to four decimal places. A properly filled can weighs 10 pounds.   \\[.1in]
\no $\diamondsuit$ What is the population? 2440 cans (NOT all cans in the world!)\\[.1in]
\no $\diamondsuit$ What is the variable of interest? Weight\\[.1in]
\no $\diamondsuit$ What is the sample? the 100 selected cans\\[.1in]
\no $\diamondsuit$ What possible inference can be made? Using $W_1, \cdots, W_{100}$, we can calculate the empirical probability: $P(W<10)$ (how?)

\foilhead[-.8in]{\textcolor{blue}{Fundamental elements of statistics (c)}}
\no  {\textcolor{magenta}{Simple random sampling:} is a sampling design where a subset of units are selected from the entire population, in such a way that all subsets of the same size are equally likely to be sampled. \\[.05in]
\no $\clubsuit$ The main benefit of simple random sampling is that {\textcolor{red}{it guarantees that the sample chosen is representative of the population.} This ensures that the statistical conclusions will be valid.\\[.05in]
\no {\textcolor{magenta}{Example:} To appreciate its customers, a company decides to get a simple random sample of 1000 each month out of all its customers and give away each selected candidate a 50 dollar gift card. \\[.05in]
\no $\heartsuit$ Sampling may cause errors: sampling errors (decrease if we enlarge sample size), and non-sampling errors (wrong scheme or inappropriate statistical techniques)\\[.05in]
\no {\textcolor{magenta}{Wrong Example (Sampling from a wrong population:)} On a market research study to determine why people like a type of soup, we sample 25 ISU students.

\foilhead[-.8in]{\textcolor{blue}{Descriptive Statistics}}
\no $\clubsuit$ Suppose we draw a random sample from {\textcolor{magenta}{a population}}; let the random variables $X_1,\cdots, X_n$ represent the sample values. The sample size is $n$. \\[.1in]
\no What properties of the population can we describe based on these? \\[.1in]
\no $\clubsuit$ Any function $W(X_1,X_2,\cdots,X_n)$ of the data is called a {\textcolor{magenta}{statistic}}.\\[.1in]
\no $\spadesuit$  We may be interested in: mean, median, quantiles, variance, standard deviation, and range.\\[.1in]
\no {\textcolor{magenta}{Population mean:} $\mu=E(X)$\\[.1in]
\no  {\textcolor{magenta}{Sample mean:}
$$\bar{X}=\frac{X_1+\cdots+X_n}n$$

\no $\diamondsuit$ Note that $\bar{X}$ is also a random variable since it is a function of $X_1,\cdots, X_n$.

\foilhead[-.8in]{\textcolor{blue}{Descriptive Statistics (Cont'd)}}
\no  $\clubsuit$ If we denote the sample values we obtained (i.e., the observations) by $x_1,x_2,\ldots,x_n$, the sample mean is 
$\bar{x}=(x_1+\cdots+x_n)/n$\\[.1in]
\no  $\spadesuit$ $\bar{X}$ will tell us something about $\mu$\\[.1in]
\no {\textcolor{magenta}{Population variance:}} $\sigma^2=\text{Var}(X)$\\[.1in]
\no {\textcolor{magenta}{Sample variance:}}
The sample variance of $X_1,\cdots, X_n$ is\\[.15in]
\hspace*{1.5in} $S^2=\frac{\sum\limits_{i=1}^n (X_i-\bar{X})^2}{n-1}=\frac{\sum\limits_{i=1}^n X_i^2-n\bar{X}^2}{n-1}$\\[.15in]
\no where $\bar{X}$ is the sample mean and $n$ is the sample size\\[.1in]
\no $\diamondsuit$ Note that $S^2$ is also a random variable since it is a function of $X_1,\cdots, X_n$.
\no Again, using the sample values, the sample variance of  $x_1,x_2,\ldots,x_n$, is \\[.1in]
\hspace*{1.5in} $s^2=\frac{\sum\limits_{i=1}^n (x_i-\bar{x})^2}{n-1}$\\[.1in]
\no  $\spadesuit$ $s^2$ tells us something about $\sigma^2$



\foilhead[-.8in]{\textcolor{blue}{Descriptive Statistics (Cont'd}}
\no  {\textcolor{magenta}{Population median:} $M$ is a number that is exceeded with probability no greater than 0.5 and is preceded with probability no greater than 0.5, i.e.
$P(X>M)\leq 0.5, \ P(X<M)\leq 0.5$\\[.1in]
\no $\clubsuit$ Comparing the mean and median, it helps to tell the shape of the distribution.
$$\text{Symmetric}\Rightarrow M=\mu$$
$$\text{Right-Skewed}\Rightarrow M<\mu$$
$$\text{Left-Skewed}\Rightarrow M>\mu$$
\begin{figure}[h]
  \centering
  \epsfig{file=f2.pdf, height=5cm}
\end{figure}
\foilhead[-.8in]{\textcolor{blue}{Descriptive Statistics (Cont'd)}}
\no $\spadesuit$ For continuous random variable, $M$ is given by $F(M)=0.5$.

\no {\textcolor{cyan}{Example:} Uniform random variable on $(a,b)$; what is the median? \\[.1in]
\no {\textcolor{cyan}{Solution:} The cdf is $F(x)=\frac{x-a}{b-a}$ for $x\in (a,b)$. Then $F(M)=0.5$ yields $M=\frac{a+b}2$, since $F(M)=1/2$ and $P(X<M)=P(X>M)=1/2$. We know $\mu=(a+b)/2$ so $\mu=M$ that implies uniform is symmetric.

\no {\textcolor{cyan}{Example:} Exponential random variable with $\lambda$; what is the median? \\[.1in]
\no {\textcolor{cyan}{Solution:} Solve $F(M)=0.5$, we have $1-e^{-\lambda x}=0.5$ so $M=\ln 2/\lambda$. Notice $\mu=E(X)=1/\lambda>M$ so, exponential is right-skewed.\\
\no {\textcolor{cyan}{Example:} Consider Bin$(5,0.5)$, then for all $x\in (2,3)$, $P(X<x)=F(2)=P(X>x)=1-F(2)=0.5$ so that any number in $(2,3)$ is a median.\\[.1in]
\no {\textcolor{magenta}{Sample median:} Sample median $\hat{M}$ is a number that is exceeded by at most a half of observations and is preceded by at most a half of observations.

\foilhead[-.8in]{\textcolor{blue}{Descriptive Statistics (Cont'd)}}
\no Denote the ordered sample (ascending) as $x_{(1)}\leq x_{(2)}\leq \cdots\leq x_{(n)}$\\[.1in]
\no {\textcolor{cyan}{Example:} Suppose the ordered sample is $1.6,2.3,3.5,4.1,5.7$; the median is clearly $3.5$; what if the sample is $1.6,2.3,3.5,4.1$ then?\\[.1in]
\no $\spadesuit$ If the sample size is odd, then the sample median is $x_{((n+1)/2)}$, if the sample size is even, then the sample median is (defined as) $(x_{(n/2)}+x_{(n/2+1)})/2$. \\[.1in]
\no $\diamondsuit$ {\textcolor{red}{If the sample size is even, the sample median is not a value in the sample!}\\[.1in]
\no {\textcolor{cyan}{Example:} $\{-1,5,8,4,-2\}$ then the ordered sample is $\{-2,-1,4,5,8\}$. So the sample median $\hat{M}=x_{((5+1)/2)}=x_{(3)}=4$. \\[.1in]
\no {\textcolor{cyan}{Example:} $\{-1,5,4,-2\}$ then the ordered sample is $\{-2,-1,4,5\}$. So $\hat{M}=(x_{(4/2)}+x_{(4/2+1)})/2=(-1+4)/2=1.5$.\\[.1in]
\no {\textcolor{magenta}{Range:} The range of a sample is $x_{(n)}-x_{(1)}$, that is the difference between maximum value and minimum value of the sample.

\foilhead[-.8in]{\textcolor{blue}{Examples:}}
\no {\textcolor{cyan}{Example 1:} The CPU time for $n=10$ randomly chosen tasks (in seconds) are
$70,36,43,49,82,48,34,62,35,15$. What is the sample mean, median, variance, and range? \\[.15in]
\no $\clubsuit$ The sample mean is $\bar{x}=\frac{x_1+\cdots +x_{10}}{10}=47.4$\\[.15in]
\no $\clubsuit$ The sample variance is $s^2=\frac{\sum\limits_{i=1}^{10} (x_i-\bar{x})^2}{10-1}
=\frac{\sum\limits_{i=1}^{10}x_i^2-10\cdot \bar{x}^2}{9}=384.0444$\\[.15in]
\no $\clubsuit$ The sample median: first order the sample as
$15, 34, 35, 36, 43, 48, 49, 62, 70, 82$ then the sample median is
$(x_{5}+x_{6})/2=45.5$.\\[.15in]
\no $\clubsuit$ The sample range is: $82-15=67$. 



\end{document}




